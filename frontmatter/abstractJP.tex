\chapter*{要旨}
%800字程度
高速道路整備の影響を正しく理解することは,その投資の必要性を判断する上で極めて重要である.特に環状高速道路は都市部の交通を環状ルートへ分散させ交通渋滞を緩和させる効果や,郊外地域で放射高速道路同士を接続し,放射・環状道路ネットワークを形成する役割が期待されている.国内における環状高速道路整備による周辺地域への効果についての実証的証拠は乏しく,ネットワーク性を考慮した効果の測定が必要である.
本研究では,放射・環状道路ネットワーク整備が周辺地域に及ぼしてきた効果の発現特性を明らかにする.具体的には.放射道路接続本数やマーケットアクセスをネットワークを考慮したアクセシビリティ指標として,放射・環状道路ネットワークが地価に及ぼす因果効果をパネルデータを用いて双方向固定効果モデルにより推定する.また,交互作用分析により,アクセシビリティが地価に与える影響に地域特性による異質性が存在するかを検討する.
