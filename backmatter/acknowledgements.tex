\chapter*{謝辞}
\addcontentsline{toc}{chapter}{謝辞}

本論文を執筆するにあたり,多くの方々にご指導やご助言をいただきました.

本論文の主査であり指導教員の織田澤利守教授には,研究の方針や結果の解釈について多くのご指導やアドバイスを賜りました.学生部屋に来られた際には,私たち学生にフランクに話しかけてくださいました.心より厚く御礼申し上げます.副査をしていただく瀬谷創准教授にも深く感謝申し上げます.中間発表では,小池教授,瀬木准教授にも貴重なご助言を賜りましたこと御礼申し上げます.

夏休み明けに私の研究が変更になるまで共同で研究をしてくださった山田さんは,私の研究が変更になった後もスライドの作り方や発表の仕方,分析結果の解釈に至るまで色々な場面でサポートしていただきました.ご自身が忙しい時でも親身に相談に乗っていただき学生生活において見本となる存在でした.本当にありがとうございました.

織田澤研究室のメンバーには,毎週のゼミだけでなくちょっとした質問の際でも多くのアドバイスをいただきました.M2の高橋さんは,研究に対してとてもストイックでゼミでも積極的に発言されていて,この研究室のリーダー的存在でした.小林さんは,私の当初の研究手法が小林さんの研究手法と関連していたので,既往研究についてアドバイスをいただきました.M1の浅田さんは就活で忙しい中,論文の内容や書き方について提出ギリギリまで相談に乗っていただきました.八杉さんは,研究に対するがとても熱量高く,私の研究に対しても私以上の知識で様々なアドバイスをいただきました.森さんは,院試の際にはご自身の膨大な勉強ノートを気前よく共有していただき,私の大学院合格の大きな助けとなりました.B4の同期の3人は,みんなおしゃべり好きでとても楽しく1年間を過ごせました.内藤さんは,院試の時も卒論の時も常に1歩私の先を行っていて自分では敵わないなと思っています.難波くんは,いつも弱気なことを言いながらも最後にはやり遂げてしまうとても優秀な人だと思っています.清谷くんは,追い込まれた時の頑張りが凄まじく,自分にはない努力する才能を持っているなと思います.留学生のShajedulさんは,私の拙い英語を一生懸命汲み取って精力的に院試勉強に取り組んでいました.また同じ学生部屋で過ごした小池研究室の皆さんや縦割りゼミなどでお世話になった計画系の皆さん,秘書の野田さんにも感謝申し上げます.

4年間の学生生活を支えてくれた両親,弟や妹にも感謝いたします.

改めて,本研究と私の大学生活に関わっていただいた皆様に,厚く御礼申し上げます.ありがとうございました.