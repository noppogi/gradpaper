\chapter*{謝辞}
\addcontentsline{toc}{chapter}{謝辞}

本研究を進めるにあたり、多くの方々からご支援とご指導をいただきました。この場を借りて、深く感謝の意を表します。

まず、神戸大学工学研究科市民工学専攻の灘研究室で学ぶ機会をいただき、社会資本投資に関する研究を進める上で多大なご指導を賜りました、指導教員の灘教授に心より感謝申し上げます。灘先生には、研究テーマの設定から分析手法の選定、さらには結果の解釈に至るまで、常に丁寧で的確なアドバイスをいただきました。特に、困難に直面した際には、適切な示唆をいただくことで研究を前進させることができました。また、研究に向き合う姿勢や学問への情熱についても、先生から多くを学ばせていただきました。

また、研究を進める過程で、研究室の先輩方や後輩たちからも多くの助言と励ましをいただきました。特に、後輩の皆さんには日々の議論や意見交換を通じて多くの刺激を受けました。研究室での時間は、学術的な成長だけでなく、人としての成長をも促してくれるものでした。これらの経験は私にとって非常に貴重であり、今後の糧として大切にしていきたいと思います。

また、研究活動の中で印象深い出来事の一つとして、研究室のメンバーと沖縄を訪れた思い出があります。このフィールドワークでは、社会資本の役割について実地で学ぶだけでなく、メンバー間の絆を深める貴重な時間を過ごしました。研究以外の場面でもともに過ごしたこれらの時間が、私の大学院生活をより充実したものにしてくれました。

さらに、家族や友人の支えにも感謝を申し上げます。研究の途中で悩みや迷いを抱えることもありましたが、日々の生活を支え、心の拠り所となってくれた家族と友人の存在があったからこそ、ここまで研究を続けることができました。

最後に、本研究に関わるすべての方々に改めて深く感謝申し上げます。この経験を糧に、今後も社会資本投資に関する研究や実践に貢献していきたいと考えています。
