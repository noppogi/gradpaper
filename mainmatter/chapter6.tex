\chapter{終わりに}

\section{本研究のまとめ}
本研究では,放射高速道路接続本数とマーケットアクセスが地価に与える影響について,2001年から2023年までの23時点のパネルデータを用いて双方向固定効果モデルにより推定した.その結果環状高速道路整備によって接続本数が1本増加すると,周辺地域の地価を0.5\%増加させることが明らかになった.また,マーケットアクセスが10\%増加すると地価は1.39\%増加することが明らかになった.

交互作用分析では,2001年の地価,最寄り駅の乗降客数,最寄り駅までの距離について,マーケットアクセスが地価に与える影響との交互作用が存在することが明らかとなった.また,今回の分析では,ICからの距離による交互作用があるとは言えず,環状高速道路の整備効果はICからの距離によらずICが存在する地域全体に発現すると考えられる.

二つの分析から環状高速道路整備によってアクセシビリティが向上すると,地価を押し上げる効果があることが明らかになったが,元々の地価や最寄りの駅についての特性がその効果の大きさに影響しており,アクセシビリティの向上が周辺地域の地価を押し上げる効果をより大きくするためには,周辺地域の特性を考慮して環状高速道路整備やIC設置を行う必要がある.
\section{今後の課題}
今後の課題は,被説明変数を従業員数や企業立地数,域内総生産とした分析が求められる.地価から直接企業の立地変化や地域の生産増大を捉えることは難しく,正確に産業に関するストック効果を捉えるためには,環状道路整備によるアクセシビリティの変化と従業員数や企業立地数,域内総生産との因果関係について分析する必要がある.

効果の発現タイミングについても分析する必要がある.交通インフラ整備には,サービス供用前に整備効果が発現するアナウンスメント効果や,サービス供用と整備効果が発現するタイミングにラグがある場合がある.特に本研究で取り上げた圏央道は段階的に整備されてきた歴史を持ち,アナウンスメント効果や整備効果発現のラグによって本来の整備効果よりも過大に推定されたり,過小に推定されている可能性がある.そのような背景を踏まえて,アナウンスメント効果や整備効果発現のラグを考慮した分析を行うべきである.