\chapter{はじめに}

\section{本研究の背景}

高速道路整備の影響を正しく理解することは,その投資の必要性を判断する上で極めて重要である.高速道路整備の効果としては,渋滞の緩和や地域間のスムーズな移動の促進などの交通市場内効果だけでなく,住民生活効果,地域経済・財政効果などの様々な交通市場外効果も期待されている.その一方で,整備効果の発現状況は非常に複雑であり,その因果関係を明らかにすることは必ずしも容易ではない.費用便益分析の実施が制度化されたことにより,事業採択時(及び再評価時)には整備効果の定量的評価が行われてきたものの,道路整備完了後に周辺地域で発現する多様な効果の計測については十分に行われてこなかった.そのような課題に対して,国土交通省では幅広い効果の把握に向けた事後評価の充実や帰着ベースの経済分析手法の活用に向けた検討などが推進されている.そのため,交通政策の事後評価は今後の我が国における道路整備を考える上で,必要不可欠である.

近年,我が国を含む多くの先進国において新たな高速道路は交通需要の少ない農村部や郊外に多く投資されている.日本の主要都市周辺では,本研究で研究対象とした関東地方における首都圏中央連絡自動車道(以下,圏央道)をはじめとして,東京外かく環状道路(外環道),東海地方における東海環状自動車道,関西地方における関西大環状道といった多くの環状高速道路が郊外地域に建設されている.こうした郊外に設置される環状高速道路は,都市部の交通を環状ルートに分散させ,都市部の交通渋滞を緩和させる効果や災害時の迂回路としての役割が期待されている.さらに環状高速道路は,都市部から郊外に向かって伸びる放射高速道路同士を接続し,放射・環状型道路ネットワークを形成する.放射・環状型道路ネットワークの形成によって,郊外地域では放射高速道路までのアクセシビリティが改善され放射高速道路を利用しての都市部や他の地域への移動が容易になる.特に産業の分野においては,他の地域への輸送の所要時間の短縮や,都市部を通行しないことで渋滞を回避して輸送を行うことが可能になり,郊外地域での企業や倉庫の立地促進や地域の生産拡大といったストック効果の発現が期待されている.また,環状高速道路が整備された地域以外でも放射・環状型道路ネットワークが拡充し,アクセシビリティが向上することでストック効果が発現すると考えられ,こうした特徴を持つ環状高速道路の整備効果を帰着ベースで測定する際には,環状高速道路の整備の有無だけでなく,放射・環状型道路のネットワーク性を考慮する必要がある.だが,国内における郊外地域への高速道路整備,特に環状高速道路整備による周辺地域への効果についての実証的証拠は乏しい.
%高速道路整備等の社会資本整備効果はフロー効果とストック効果に分けられる.フロー効果は公共投資によって,生産,雇用及び所得等の経済活動が波及的に創出され,短期的に経済全体を拡大させる効果である.ストック効果は整備された社会資本が機能を発揮することによって,整備直後から継続的に中長期にわたって得られる効果である.\cite{kokudo_stock}特にストック効果を最大化することが現在の社会資本整備において重要視されている.
%図\ref{fig:road_investment},図\ref{fig:road_effect}に示すように
%我が国では以前より,継続的に高速道路ネットワーク整備が実施されており,高速道路ネットワークの整備に伴って各地域間で移動時間が短縮されるなど様々な恩恵がもたらされてきた.そして近年,主要都市周辺では,本研究で研究対象とした関東地方における首都圏中央連絡自動車道(以下,圏央道)をはじめとして,東京外かく環状道路(外環道),東海地方における東海環状自動車道,関西地方における関西大環状道といった多くの環状道路が郊外地域に建設されている.こうした郊外に設置される環状高速道路は,都市部を通過する交通を環状ルートへ分散させることで,都市部の交通渋滞を緩和させるといった効果は勿論,放射道路間の高速道路ネットワークの構築や災害時等には迂回路としての役割が期待されている.その上,そのような交通市場内効果だけでなく高速道路の建設によってその周辺地域では,住民生活効果,地域経済・財政効果など,様々な交通市場外効果を受け取ることができる.地域のアクセシビリティの向上に伴う生産力の拡大,人口や雇用の増加,公共サービスの向上,そして郊外地域では特に.住宅地・産業地としての魅力が増すことによる土地資産価値の上昇が期待されている.
\section{本研究の目的と手法}

本研究は,放射・環状型道路ネットワークが周辺地域に及ぼす効果の発現特性を明らかにすることを目的としている.本研究では,首都圏郊外の圏央道をケーススタディとして取り上げ,圏央道整備によってアクセシビリティがどのように変化するかについてとアクセシビリティの変化がその周辺地域に与えた影響について双方向固定効果モデルを用いて分析を行う.

第一に,GIS形式の高速道路時系列データを用いて,2001年から2023年までの各時点において圏央道の各インターチェンジ(以下IC)の圏央道を通じて接続する放射高速道路の本数を調査し,接続本数データを作成する.次にNITASを用いて各時点の各ICから放射高速道路までの経路探索を行い,得られた所要時間データを使用してマーケットアクセスを作成する.こうして得られた接続本数データとマーケットアクセスデータをアクセシビリティの指標とし,2001年から2023年までの23時点の地価データを用いてアクセシビリティの変化が地価に及ぼす因果効果を推定する.また,発展分析として交互作用分析を行う.

\section{本研究の構成}

本研究は全6章で構成されている.第2章では,これまでの高速道路整備効果の実証的研究や,高速道路のアクセシビリティについての既往研究を整理し,本研究の新規性を明確にする.第3章では,本研究で使用したデータの説明と本研究におけるマーケットアクセスの定義と経路探索の方法について述べる.第4章では,本研究の基本分析と発展分析の枠組みについて説明を行う.第5章では,基本分析と発展分析の結果を示し,その考察を行う.最後に,第6章では,本研究の結論と今後の課題について述べる.

