\chapter{はじめに}

\section{本研究の背景}

高速道路整備の影響を正しく理解することは,その投資の必要性を判断する上で極めて重要である.高速道路開発計画は適切な評価によって過不足なく正当化することができ,高速道路整備の効果については,渋滞の緩和や地域間のスムーズな移動の促進などの交通市場内効果だけでなく,住民生活効果,地域経済・財政効果などの様々な交通市場外効果も期待されている.しかしその一方で,整備効果の発現状況は非常に複雑であり,その因果関係を明らかにすることは必ずしも容易ではない.費用便益分析の実施が制度化されたことにより,事業採択時(及び再評価時)には整備効果の定量的評価が行われてきたものの,道路整備完了後に周辺地域で発現する多様な効果の計測については十分に行われてこなかった.そのような課題に対して,国土交通省では幅広い効果の把握に向けた事後評価の充実や帰着ベースの経済分析手法の活用に向けた検討などが推進されている.そのため,交通政策の事後評価は今後の我が国における道路整備を考える上で,必要不可欠である.

また近年,我が国を含む多くの先進国では新たな高速道路は交通需要の少ない農村部や郊外に多く投資されている.日本の主要都市周辺では,本研究で研究対象とした関東地方における首都圏中央連絡自動車道(以下,圏央道)をはじめとして,東京外かく環状道路(外環道),東海地方における東海環状自動車道,関西地方における関西大環状道といった多くの環状道路が郊外地域に建設されている.こうした郊外に設置される環状高速道路は,都市部を通過する交通を環状ルートへ分散させることで,都市部の交通渋滞を緩和させるといった効果は勿論,放射道路間の高速道路ネットワークの構築や災害時等には迂回路としての役割が期待されている.さらに,交通市場外効果として,地域のアクセス性向上に伴う生産力の拡大,人口や雇用の増加,公共サービスの向上,住宅地・産業地としての魅力向上による土地資産価値上昇が期待されている.今後より郊外地域への高速道路投資を進めていくためには,交通需要の少ない地域への高速道路投資の正当化が必要であり,そのためには高速道路整備効果についてより明確な証拠が強く求められている.だが,国内における郊外地域への高速道路整備,特に環状高速道路整備による周辺地域への効果についての実証的証拠は乏しいのが現状である.
%高速道路整備等の社会資本整備効果はフロー効果とストック効果に分けられる.フロー効果は公共投資によって,生産,雇用及び所得等の経済活動が波及的に創出され,短期的に経済全体を拡大させる効果である.ストック効果は整備された社会資本が機能を発揮することによって,整備直後から継続的に中長期にわたって得られる効果である.\cite{kokudo_stock}特にストック効果を最大化することが現在の社会資本整備において重要視されている.
%図\ref{fig:road_investment},図\ref{fig:road_effect}に示すように
%我が国では以前より,継続的に高速道路ネットワーク整備が実施されており,高速道路ネットワークの整備に伴って各地域間で移動時間が短縮されるなど様々な恩恵がもたらされてきた.そして近年,主要都市周辺では,本研究で研究対象とした関東地方における首都圏中央連絡自動車道(以下,圏央道)をはじめとして,東京外かく環状道路(外環道),東海地方における東海環状自動車道,関西地方における関西大環状道といった多くの環状道路が郊外地域に建設されている.こうした郊外に設置される環状高速道路は,都市部を通過する交通を環状ルートへ分散させることで,都市部の交通渋滞を緩和させるといった効果は勿論,放射道路間の高速道路ネットワークの構築や災害時等には迂回路としての役割が期待されている.その上,そのような交通市場内効果だけでなく高速道路の建設によってその周辺地域では,住民生活効果,地域経済・財政効果など,様々な交通市場外効果を受け取ることができる.地域のアクセシビリティの向上に伴う生産力の拡大,人口や雇用の増加,公共サービスの向上,そして郊外地域では特に.住宅地・産業地としての魅力が増すことによる土地資産価値の上昇が期待されている.
\section{本研究の目的と手法}

本研究は,放射・環状道路ネットワークが周辺地域に及ぼす効果の発現特性を明らかにすることを目的としている.本研究では,首都圏郊外の圏央道をケーススタディとして取り上げ,圏央道整備によってアクセシビリティがどのように変化するかと,アクセシビリティの変化がその周辺地域に与えた影響について地価に着目して双方向固定効果モデルを用いて分析を行う,

まず,GIS形式の高速道路時系列データを用いて,各インターチェンジ(以下IC)の2001年から2023年までの各時点において何本の放射道路と圏央道を通じて接続しているかを調査し,接続本数データを作成する.次にNITASを用いて各年次において各ICから放射道路まで経路探索を行い,そこで得られた所要時間データを使用してマーケットアクセス指標を作成する.こうして得られた接続本数データとマーケットアクセス指標データをアクセシビリティの指標とし,2001年から2023年までの23時点の地価データを用いてアクセシビリティの変化が地価に及ぼす因果効果を推定する.また,観測点のICからの距離や2001年時点での地価の違いによる個別トレンドが存在しているかについても分析を行う.

\section{本研究の構成}

本研究は全6章で構成されている.第1章「はじめに」では,高速道路整備効果の評価の重要性と環状高速道路の特徴を説明し,その評価の課題と研究目的を提示する.第2章「本研究の位置付け」では,これまでの高速道路整備効果の実証的研究や,高速道路のアクセしビィティについての既往研究を整理し,本研究の新規性を明確にする.第3章「分析データ」では,本研究で使用したデータの説明と作成したデータの作成手法,本研究におけるマーケットアクセス指標の定義と経路探索の方法について述べる..第4章「分析」では,本研究の基本分析と個別トレンド分析の枠組みについて説明を行う.第5章「分析結果と考察」では,基本分析と個別トレンド分析の結果を示し,その考察を行う.最後に,第6章「終わりに」では,本研究の結論と今後の課題について述べる.

