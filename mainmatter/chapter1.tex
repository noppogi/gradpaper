\chapter{はじめに}

\section{本研究の背景}

高速道路整備の影響を正しく理解することは,その投資の必要性を判断する上で極めて重要である.高速道路開発計画は適切な評価によって過不足なく正当化することができ,高速道路整備の効果については,渋滞の緩和や地域間のスムーズな移動の促進などの交通市場内効果だけでなく,住民生活効果,地域経済・財政効果などの様々な交通市場外効果も期待されている.しかしその一方で,整備効果の発現状況は非常に複雑であり,その因果関係を明らかにすることは必ずしも容易ではない.費用便益分析の実施が制度化されたことにより,事業採択時(及び再評価時)には整備効果の定量的評価が行われてきたものの,道路整備完了後に周辺地域で発現する多様な効果の計測については十分に行われてこなかった.そのような課題に対して,国土交通省では幅広い効果の把握に向けた事後評価の充実や帰着ベースの経済分析手法の活用に向けた検討などが推進されている.そのため,交通政策の事後評価は今後の我が国における道路整備を考える上で,必要不可欠である.

また近年,我が国を含む多くの先進国では新たな高速道路は交通需要の少ない農村部や郊外に多く投資されている.日本の主要都市周辺では,本研究で研究対象とした関東地方における首都圏中央連絡自動車道(以下,圏央道)をはじめとして,東京外かく環状道路(外環道),東海地方における東海環状自動車道,関西地方における関西大環状道といった多くの環状道路が郊外地域に建設されている.こうした郊外に設置される環状高速道路は,都市部を通過する交通を環状ルートへ分散させることで,都市部の交通渋滞を緩和させるといった効果は勿論,放射道路間の高速道路ネットワークの構築や災害時等には迂回路としての役割が期待されている.さらに,交通市場外効果として,地域のアクセス性向上に伴う生産力の拡大,人口や雇用の増加,公共サービスの向上,住宅地・産業地としての魅力向上による土地資産価値上昇が期待されている.今後より郊外地域への高速道路投資を進めていくためには,交通需要の少ない地域への高速道路投資の正当化が必要であり,そのためには高速道路整備効果についてより明確な証拠が強く求められている.だが,国内の郊外地域へにおける高速道路整備,特に環状高速道路整備による周辺地域への効果についての実証的証拠は乏しいのが現状である.
%高速道路整備等の社会資本整備効果はフロー効果とストック効果に分けられる.フロー効果は公共投資によって,生産,雇用及び所得等の経済活動が波及的に創出され,短期的に経済全体を拡大させる効果である.ストック効果は整備された社会資本が機能を発揮することによって,整備直後から継続的に中長期にわたって得られる効果である.\cite{kokudo_stock}特にストック効果を最大化することが現在の社会資本整備において重要視されている.
%図\ref{fig:road_investment},図\ref{fig:road_effect}に示すように
%我が国では以前より,継続的に高速道路ネットワーク整備が実施されており,高速道路ネットワークの整備に伴って各地域間で移動時間が短縮されるなど様々な恩恵がもたらされてきた.そして近年,主要都市周辺では,本研究で研究対象とした関東地方における首都圏中央連絡自動車道(以下,圏央道)をはじめとして,東京外かく環状道路(外環道),東海地方における東海環状自動車道,関西地方における関西大環状道といった多くの環状道路が郊外地域に建設されている.こうした郊外に設置される環状高速道路は,都市部を通過する交通を環状ルートへ分散させることで,都市部の交通渋滞を緩和させるといった効果は勿論,放射道路間の高速道路ネットワークの構築や災害時等には迂回路としての役割が期待されている.その上,そのような交通市場内効果だけでなく高速道路の建設によってその周辺地域では,住民生活効果,地域経済・財政効果など,様々な交通市場外効果を受け取ることができる.地域のアクセシビリティの向上に伴う生産力の拡大,人口や雇用の増加,公共サービスの向上,そして郊外地域では特に.住宅地・産業地としての魅力が増すことによる土地資産価値の上昇が期待されている.

\begin{figure}[H]
  \centering
  \includegraphics[width=10cm]{/Users/sawadaryou/Documents/gradpaper/figure/road_investment.jpeg}
  \caption{道路投資の主な効果とその分類}
  \label{fig:road_investment}
\end{figure}
\begin{figure}[H]
  \centering
  \includegraphics[width=10cm]{figure/road_effect.jpeg}
  \caption{道路整備による効果の項目体系}
  \label{fig:road_effect}
\end{figure}
\section{本研究の目的と手法}

本研究の目的は,社会資本投資が地域経済に及ぼす影響を実証的に明らかにし,その波及効果を定量的に評価することである.具体的には,社会資本投資が地域の雇用創出や所得向上,生産性向上にどのような形で寄与するかを検証し,政策効果を科学的に測定することを目指す.特に,地域特性や経済的条件の違いが投資効果に与える影響を詳細に分析することで,投資の効率性を高める方策を提案する.また,短期的な経済効果だけでなく,長期的な持続可能性や地域間格差への影響についても検討を行い,政策立案に資する実践的な知見を提供する.本研究は,社会資本投資を通じた地域経済の発展メカニズムを明らかにし,今後のインフラ投資のあり方や地域政策の方向性に対する新たな視点を提示することを目指す.

\section{本研究の構成}

本研究は全5章で構成されている.第1章「はじめに」では,社会資本投資が地域経済に与える影響の重要性を背景として説明し,研究目的と課題を提示する.第2章「既往研究の整理と本研究の位置付け」では,社会資本投資と経済成長,全要素生産性(TFP)に関する理論や実証研究を概観し,本研究の新規性を明確にする.第3章「研究方法」では,地域間の付加価値成長率とTFPを従属変数とし,社会資本投資を主要な独立変数として回帰分析を実施する手法を説明する.さらに,使用するデータ(例: 地域経済データ,インフラ投資統計)やマクロ経済モデルの枠組みを具体的に提示する.第4章「結果と考察」では,回帰分析の結果を詳細に報告し,社会資本投資が地域経済成長と生産性向上に与える影響を定量的に評価する.また,結果に基づいた政策的インプリケーションを議論する.最後に,第5章「おわりに」では,本研究の総括と得られた知見を示すとともに,今後の課題や研究展望を提示する.

