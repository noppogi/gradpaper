\chapter{分析対象とデータの整備}

\section{分析対象}
本研究では,社会資本投資が地域経済に与える影響を実証的に分析するため,地域間の付加価値成長率および全要素生産性(TFP)を従属変数とした回帰分析を行う.独立変数としては,社会資本投資額や社会資本ストック指標を中心に,各地域の経済特性を反映する補足変数(人口密度,産業構造,教育水準など)を組み込むことで,投資効果の多面的な評価を可能にする.

分析には,全国規模の統計データや地域別の経済データ(例:内閣府経済統計,国土交通省の社会資本整備データ)を使用する.さらに,パネルデータ分析を採用することで,時間的変化を考慮しつつ,個別地域間の固定効果を制御する.これにより,各地域での社会資本投資が経済成長や生産性に及ぼす直接的および間接的な効果を明確化することを目指す.

また,マクロ経済モデルを補完的に用い,社会資本投資が経済成長に寄与するメカニズムを理論的に検証する.このモデルは,生産関数に社会資本を組み込むことで,全体的な経済構造への影響を評価する枠組みを提供する.さらに,投資効果が短期的な需要促進にとどまらず,長期的な供給力強化や地域間格差是正にどのように寄与するかを定量的に分析する.

本研究の手法は,従来の社会資本投資研究における知見を深化させるとともに,地域特性や投資規模による効果の違いを明確にする点で新規性を有する.

\section{データの整備}

本研究では,地域経済における社会資本投資の影響を以下の数理モデルで定式化する.

\subsection{経路探索}

地域 \( i \) における生産量 \( Y_i \) は次のように定義する:

\[
	Y_i = A_i \cdot K_i^\alpha \cdot L_i^\beta \cdot G_i^\gamma
\]

ここで:
\begin{itemize}
	\renewcommand{\labelitemi}{}
	\item \( Y_i \) は地域 \( i \) の生産量(付加価値など)
	\item \( A_i \) は全要素生産性(TFP)
	\item \( K_i \) は民間資本ストック
	\item \( L_i \) は労働投入量
	\item \( G_i \) は社会資本ストック
	\item \( \alpha, \beta, \gamma \) は資本,労働,社会資本の生産弾力性(\( \alpha + \beta + \gamma \leq 1 \))
\end{itemize}

\subsection{マーケットアクセスの算出}

社会資本ストック \( G_i \) は,時間 \( t \) における社会資本投資 \( I_{G,i}(t) \) によって次のように変化する:

\[
	G_i(t+1) = (1-\delta) \cdot G_i(t) + I_{G,i}(t)
\]

ここで:
\begin{itemize}
	\renewcommand{\labelitemi}{}
	\item \( \delta \) は社会資本の減耗率(\( 0 < \delta < 1 \))
	\item \( I_{G,i}(t) \) は時点 \( t \) における社会資本投資
\end{itemize}




\section{指標と記述統計}
\vskip\baselineskip
\input{table/table_sample1.tex}

本研究では,社会資本投資が地域経済に与える影響を分析するため,主に以下のデータを使用する.社会資本投資に関するデータは,国土交通省が提供する社会資本ストック統計および社会資本投資額の統計を利用し,地域ごとの社会資本整備の状況を把握する.地域経済の付加価値成長率や全要素生産性(TFP)の計算には,内閣府が提供する地域別経済統計を活用する.加えて,人口密度や産業構造,教育水準といった地域特性を表すデータは,総務省の「国勢調査」や「経済センサス」から取得する.

表 \ref{tab:data} に使用するデータの詳細を示す.



