\chapter{分析データと集計}

\section{分析対象}
\subsection{対象の環状道路と放射道路}
本研究では,圏央道整備による整備効果の推定を行なう.
圏央道(首都圏中央連絡自動車道)は,東京都心から約40〜60kmを環状に連絡する高規格幹線道路である.東名高速道路,中央自動車道,関越自動車道,東北縦貫自動車道,常磐自動車道,東関東自動車道等の放射状に延びる高速道路や都心郊外の主要都市を連絡し、東京湾アクアライン、東京外かく環状道路等と一体となって首都圏の広域的な幹線道路網を形成している\cite{kokudo2024}.最初の区間は1996年に開通し,その後徐々に延長され,最新区間は2017年に開通した.計画総延長は約300kmで,すでに約270kmの主要区間が開通しているほか,南西部や東部では2026年の開通が予定されている.ほとんどの区間が4車線区間であるが,東部の交通量が少ない一部区間では,暫定2車線区間として建設されており,車線拡張プロジェクトが進行中である.圏央道の概要については図\ref{fig:ken-o-express}に,圏央道の各区間供用年次については表\ref{圏央道供用年次}に示す.

また,本研究での放射道路は圏央道に接続する東名高速道路,中央自動車道,関越自動車道,東北縦貫自動車道,常磐自動車道,東関東自動車道水戸線,東関東自動車道館山線を指す.

\begin{figure}[H]
  \centering
  \includegraphics[width=10cm]{figure/ken-o-express.png}
  \caption{圏央道概要図}
  \label{fig:ken-o-express}
\end{figure}
\newpage
\begin{table}[h]
  \begin{center}
    \caption{圏央道供用年次}
    \vspace{1.5zw}
    \begin{tabular}{|c|p{12cm}|}\hline \label{圏央道供用年次}
      年 & ~~~~~~~~~~~~~~~~~~~~~~~~~~~~~~~~~~~~~~~~区間供用年次 \\ \hline \hline
      〜2001 & 茅ヶ崎JCT〜藤沢IC(1998),青梅IC〜鶴ヶ島JCT(1996),東金IC〜松尾横芝IC(1998) \\ \hline
      2002 & 日の出IC〜青梅IC \\ \hline
      2003 & つくばJCT〜つくば牛久IC \\ \hline
      2004 & \multicolumn{1}{|c|}{------} \\ \hline
      2005 & あきる野IC〜日の出IC \\ \hline
      2006 & \multicolumn{1}{|c|}{------} \\ \hline
      2007 & つくば牛久IC〜阿見東IC,木更津東IC〜木更津JCT,八王子JCT〜あきる野IC(中央道〜関越道路) \\ \hline
      2008 & 鶴ヶ島JCT〜川島IC \\ \hline
      2009 & 阿見東IC〜稲敷IC \\ \hline
      2010 & 海老名JCT〜海老名IC,川島IC〜桶川河本IC,つくば中央IC〜つくばJCT \\ \hline
      2011 & 白岡菖蒲IC〜久喜白岡JCT \\ \hline
      2012 & 高尾山IC〜八王子JCT \\ \hline
      2013 & 海老名IC〜相模原愛川IC,茅ヶ崎JCT〜寒川北IC,東金IC〜木更津東IC \\ \hline
      2014 & 稲敷IC〜神崎IC,相模原愛川IC〜高尾山IC(東名〜中央道連絡) \\ \hline
      2015 & 寒川北IC〜海老名JCT(新湘南BP〜東名連絡),久喜白岡JCT〜境古河IC,神崎IC〜大栄JCT(常磐道〜東関道連絡),桶川北本IC〜白岡菖蒲IC(関越〜東北道連絡) \\ \hline
      2016 & \multicolumn{1}{|c|}{------} \\ \hline
      2017 & 境古河IC〜つくば中央IC(東北道〜常磐道連絡) \\ \hline
      今後 & 大栄JCT〜松尾横芝IC(2025 or 2026 予定)\\ \hline
      & 藤沢IC〜釜利谷JCT(未定) \\ \hline
    \end{tabular}
  \end{center}
\end{table}

\subsection{分析対象地域}
本研究の対象地域は,東京都(諸島部を除く),神奈川県,埼玉県,千葉県,茨城県,かつ,各圏央道のインターチェンジからの距離が10km以内の地点である.圏央道から離れた地点を分析対象から除いた理由として,都心に近い地点では都心からのよる様々な影響を受けやすく,正確な処置効果を算出するのに偏りを生じさせる可能性が高いと考えたからである.これについて圏央道の整備効果について分析を行なったWetwitooら(2024)\cite{Weteitoo2024}の研究においても,東京駅までの距離が30kmより近い地点を分析対象に入れることで有意性が低くなることを明らかにしている.また,圏央道の外側の部分については,遠く離れた地点は圏央道の影響を受けているとは到底考えにくく,処置効果を適切に算出できないと考えたため,対象から除外した.

\section{使用データ}
\subsection{高速道路データ}
高速道路時系列データはGIS形式の点ベースのIC情報と線ベースの路線情報で構成されており、IC情報には供用開始年などの情報が含まれている。

\subsection{地価公示データ}
地価公示データは国土交通省の土地鑑定委員会が毎年1月1日に各観測点の1平方メートルあたりの鑑定地価である.交通政策の整備効果を地価の動きから調査する際には鑑定バイアスのない市場取引価格を用いるのが最善である.しかしながらプライバシー保護の観点などからそのようなデータは一般に公開されていない.また,そうしたデータは分析に用いるにはサンプル数が少ないといった問題もある.そこで,本研究では鑑定価格の情報を提供する国土数値情報の地価公示データを採用している.\cite{shimizu2006}\cite{kunimi2021}この地価公示データにおける地価変動は実際の市場取引価格の変動に比べると穏やかである.そのため,本研究による分析結果は圏央道整備による整備効果の最小値を示していると考えられる。
地価観測点はしばしば変更されている.また,鉄道駅の開通などに伴って新たな地点が追加される場合もある.そのため,本研究では分析対象期間である2001~2023年の間で継続して地価が観測されている地点を対象として分析を行っている.また,高速道路時系列データ,地価公示データから各観測点の最寄りICをGIS上で算出し,地価データに最寄りIC情報を含めている.
\subsection{所要時間データ}
\subsubsection{NITAS}
NITASは国土交通省が提供している道路・鉄道・航空・船舶の各交通機関を組み合わせたモード横断的な観点で交通体系の分析が可能なシステムである.複数対複数の地点間による経路探索や,地図上に探索結果等っを図化することができ,不通過区間を設定することで現在の交通ネットワークだけでなく,過去や将来の交通ネットワークでも経路探索が可能である.\cite{NITAS}
\subsubsection{経路探索条件}
所要時間データはNITASを用いて,交通手段として有料道路と一般道路のみを使用し所要時間最小の条件で経路探索を行って収集した.一般には所要時間最小ではなく一般化費用を最小とする条件で経路探索を行うが,所要時間を最小とすることで本研究のマーケットアクセスは企業の立地ポテンシャルを表し,産業目線での分析が可能になっていると言える.
\section{データの集計}
\subsection{集計対象}
\subsection{データの加工}
\subsubsection{接続本数データ}
\subsubsection{マーケットアクセス指標}
本研究では織田澤・足立らを参考に,放射・環状道路ネットワークを考慮したn年におけるマーケットアクセス指標を次のように定義する.
\[MA^{n} = \sum_{i \neq j}{t_{j}/c^{n}_{ij}}\]
ここで,$t_{j}$は放射道路jの上下線日交通量の合計であり,令和3年度交通センサスより収集した.$c^{n}_{ij}$はn年におけるインターチェンジiと放射道路jの間の自動車による所要時間である.放射道路の行き先の人口や雇用者密度などの経済規模ではなく放射道路の交通量を用いたのは,本研究が放射・環状道路ネットワーク整備の効果の抽出を目的としているためである.交通量を放射道路の価値と考えることで,放射道路の行き先の経済規模の影響を排除し放射・環状ネットワークの接続性を示す指標としている.n年におけるインターチェンジiと放射道路j間の所要時間は国土交通省が提供しているNITASを用いて2001〜2023年の各年度において経路探索を行い算出した.具体的な経路探索の方法については次節で述べる.こうして求めたマーケットアクセスを地価データの最寄りインターチェンジの情報を用いて地価観測点と紐づけた.

\section{集計データの記述統計}








\section{指標と記述統計}
\vskip\baselineskip
\input{table/table_sample1.tex}

本研究では,社会資本投資が地域経済に与える影響を分析するため,主に以下のデータを使用する.社会資本投資に関するデータは,国土交通省が提供する社会資本ストック統計および社会資本投資額の統計を利用し,地域ごとの社会資本整備の状況を把握する.地域経済の付加価値成長率や全要素生産性(TFP)の計算には,内閣府が提供する地域別経済統計を活用する.加えて,人口密度や産業構造,教育水準といった地域特性を表すデータは,総務省の「国勢調査」や「経済センサス」から取得する.

表 \ref{tab:data} に使用するデータの詳細を示す.



