\chapter{分析データ}

\section{分析対象}
\subsection{対象の環状道路と放射道路}
本研究では,圏央道整備による整備効果の推定を行なう.
圏央道(首都圏中央連絡自動車道)は,東京都心から約40〜60kmを環状に連絡する高規格幹線道路である.東名高速道路,中央自動車道,関越自動車道,東北縦貫自動車道,常磐自動車道,東関東自動車道等の放射状に延びる高速道路や都心郊外の主要都市を連絡し、東京湾アクアライン、東京外かく環状道路等と一体となって首都圏の広域的な幹線道路網を形成している\cite{kokudo2024}.最初の区間は1996年に開通し,その後徐々に延長され,最新区間は2017年に開通した.計画総延長は約300kmで,すでに約270kmの主要区間が開通しているほか,南西部や東部では2026年の開通が予定されている.ほとんどの区間が4車線区間であるが,東部の交通量が少ない一部区間では,暫定2車線区間として建設されており,車線拡張プロジェクトが進行中である.圏央道の概要については図\ref{fig:ken-o-express}に,圏央道の各区間供用年次については表\ref{圏央道供用年次}に示す.

本研究での放射高速道路は圏央道に接続する東名高速道路,中央自動車道,関越自動車道,東北縦貫自動車道,常磐自動車道,東関東自動車道水戸線,東関東自動車道館山線を指す.

\begin{figure}[H]
  \centering
  \includegraphics[width=10cm]{figure/ken-o-express.png}
  \caption{圏央道概要図}
  \label{fig:ken-o-express}
\end{figure}
\newpage
\begin{table}[h]
  \begin{center}
    \caption{圏央道供用年次}
    \vspace{1.5zw}
    \begin{tabular}{|c|p{12cm}|}\hline \label{圏央道供用年次}
      年 & ~~~~~~~~~~~~~~~~~~~~~~~~~~~~~~~~~~~~~~~~区間供用年次 \\ \hline \hline
      〜2001 & 茅ヶ崎JCT〜藤沢IC(1998),青梅IC〜鶴ヶ島JCT(1996),東金IC〜松尾横芝IC(1998) \\ \hline
      2002 & 日の出IC〜青梅IC \\ \hline
      2003 & つくばJCT〜つくば牛久IC \\ \hline
      2004 & \multicolumn{1}{|c|}{------} \\ \hline
      2005 & あきる野IC〜日の出IC \\ \hline
      2006 & \multicolumn{1}{|c|}{------} \\ \hline
      2007 & つくば牛久IC〜阿見東IC,木更津東IC〜木更津JCT,八王子JCT〜あきる野IC(中央道〜関越道路) \\ \hline
      2008 & 鶴ヶ島JCT〜川島IC \\ \hline
      2009 & 阿見東IC〜稲敷IC \\ \hline
      2010 & 海老名JCT〜海老名IC,川島IC〜桶川河本IC,つくば中央IC〜つくばJCT \\ \hline
      2011 & 白岡菖蒲IC〜久喜白岡JCT \\ \hline
      2012 & 高尾山IC〜八王子JCT \\ \hline
      2013 & 海老名IC〜相模原愛川IC,茅ヶ崎JCT〜寒川北IC,東金IC〜木更津東IC \\ \hline
      2014 & 稲敷IC〜神崎IC,相模原愛川IC〜高尾山IC(東名〜中央道連絡) \\ \hline
      2015 & 寒川北IC〜海老名JCT(新湘南BP〜東名連絡),久喜白岡JCT〜境古河IC,神崎IC〜大栄JCT(常磐道〜東関道連絡),桶川北本IC〜白岡菖蒲IC(関越〜東北道連絡) \\ \hline
      2016 & \multicolumn{1}{|c|}{------} \\ \hline
      2017 & 境古河IC〜つくば中央IC(東北道〜常磐道連絡) \\ \hline
      今後 & 大栄JCT〜松尾横芝IC(2025 or 2026 予定)\\ \hline
      & 藤沢IC〜釜利谷JCT(未定) \\ \hline
    \end{tabular}
  \end{center}
\end{table}

\subsection{分析対象地域}
本研究の分析対象地域は,東京都(諸島部を除く),神奈川県,埼玉県,千葉県,茨城県,かつ,各圏央道のICからの距離が10km以内の地点である.図に分析対象地域を示す.圏央道から離れた地点を分析対象から除いた理由として,都心に近い地点では都心からのよる様々な影響を受けやすく,正確な処置効果を算出するのに偏りを生じさせる可能性が高いと考えたからである.これについて圏央道の整備効果について分析を行なったWetwitooら(2024)\cite{Weteitoo2024}の研究においても,東京駅までの距離が30kmより近い地点を分析対象地域に入れることで有意性が低くなることを明らかにしている.また,圏央道の外側の部分については,遠く離れた地点は圏央道の影響を受けているとは到底考えにくく,処置効果を適切に算出できないと考えたため,対象から除外した.

\section{使用データ}
\begin{table}[h!]
  \centering
  \renewcommand{\arraystretch}{1.2} % 行間を広げる
  \begin{tabularx}{\textwidth}{p{0.20\textwidth} p{0.25\textwidth} p{0.25\textwidth} X}
  \toprule
  \textbf{データ分類}     &\textbf{データ項目}         & \textbf{内容}                                    & \textbf{出典}                   \\
  \midrule
  高速道路データ          &高速道路時系列データ         & GIS形式の点ベースのIC情報と線ベースの路線情報           & 国土交通省「国土数値情報」    \\
  地価データ             &地価公示データ              & 各地価観測点の鑑定地価                                & 国土交通省「国土数値情報」    \\
  交通量データ            &上下線合計日交通量           & 各区間の24時間交通量の上下線合計                       & 全国道路・街頭交通情勢調査              \\
  所要時間データ          &総所要時間                  & 各ICから各放射道路のJCTまでの所要時間                   & NITAS            \\
  乗降客数データ          &駅別乗降客数データ           &駅別の乗降客数                                         &国土交通省「国土数値情報」\\
  \bottomrule
  \end{tabularx}
  \caption{使用するデータ一覧}
  \label{data_list}
  \end{table}
  


\subsection{高速道路データ}
高速道路データは国土数値情報の高速道路時系列データを使用した.
高速道路時系列データはGIS形式の点ベースのIC情報と線ベースの路線情報で構成されており,IC情報には供用開始年などの情報が含まれている.

\subsection{地価データ}
東京都,神奈川県,埼玉県,千葉県,茨城県の地価データは国土数値情報の地価公示データ(2001年〜2023年分)をGIS形式で使用した,
地価公示データは国土交通省の土地鑑定委員会が毎年1月1日に各観測点の1平方メートルあたりの鑑定地価である.交通政策の整備効果を地価の動きから調査する際には鑑定バイアスのない市場取引価格を用いるのが最善であるが,プライバシー保護の観点などからそのようなデータは一般に公開されていない.また,そうしたデータは分析に用いるにはサンプル数が少ないといった問題もある.そこで,本研究では鑑定価格の情報を提供する国土数値情報の地価公示データを採用している.\cite{shimizu2006}\cite{kunimi2021}この地価公示データにおける地価変動は実際の市場取引価格の変動に比べると穏やかである.そのため,本研究による分析結果は圏央道整備による整備効果の最小値を示していると考えられる。

地価観測点はしばしば変更されており,鉄道駅の開通などに伴って新たな地点が追加される場合もある.そのため,本研究では分析対象期間である2001年~2023年の間で継続して地価が観測されている地点を対象として分析を行っており,対象の観測点は,890点である.また,高速道路時系列データ,地価公示データから各観測点の最寄り圏央道ICをGIS上で算出し,地価データに最寄り圏央道IC情報を含めている.また,各観測点のも良い駅もGIS上算出し,地価データに最寄り駅情報を含めている.

高速道路データと対象の地価観測点を地図上にプロットしたのが図\ref{plot_landprice}である.オレンジのポイントが対象の地価観測点,水色のポイントは圏央道IC,緑色の線が現在開通済みの圏央道路線,青色の線が放射高速道路(東名高速道路,中央自動車道,関越自動車道,東北縦貫自動車道,常磐自動車道,東関東自動車道水戸線,東関東自動車道館山線)を表している.
\begin{figure}[H]
  \centering
  \includegraphics[width=9cm]{figure/plot_landprice2.png}
  \caption{対象地価観測点-全体プロット図}
  \label{plot_landprice}
\end{figure}

\subsection{交通量データ}
各放射高速道路の交通量のデータとして2021年実施分の道路交通センサスから上下線合計日交通量(台)を使用した.各放射高速道路の圏央道とのJCTから放射高速道路の下り方面の一つ先のICまでの区間の上下線合計交通量を放射道路の交通量としている.表\ref{radiationroad_number}に各放射高速道路の交通量とその計測区間を示す.
\begin{table}[h!]
  \centering
  \renewcommand{\arraystretch}{1.2} % 行間を広げる
  \begin{tabularx}{\textwidth}{p{0.25\textwidth}  p{0.30\textwidth} X}
  \toprule
  \textbf{放射道路名}     &\textbf{上下線合計日交通量(台)}         & \textbf{計測区間名}                         \\
  \midrule
  東名高速道路            &137337                               & 海老名JCT〜厚木IC   \\
  中央自動車道            &50040                                & 八王子JCT〜相模湖東IC \\
  関越自動車道            &95913                                & 鶴ヶ島JCT〜鶴ヶ島IC  \\
  東北縦貫自動車          &86328                                & 久喜白岡JCT〜久喜IC  \\
  常磐自動車道            &52557                                & つくばJCT〜桜土浦IC\\
  東関東自動車道水戸線     &22367                                & 大栄JCT〜大栄IC\\
  東関東自動車道館山線     &28029                                & 木更津JCT〜木更津南IC\\
  \bottomrule
  \end{tabularx}
  \caption{放射道路の交通量}
  \label{radiationroad_number}
  \end{table}
  



\subsection{所要時間データ}
所要時間データは国土交通省が提供するNITASを用いて圏央道の各ICから各放射高速道路までの経路探索を行い,収集した.

\subsubsection{NITAS}
NITASは国土交通省が提供している道路・鉄道・航空・船舶の各交通機関を組み合わせたモード横断的な観点で交通体系の分析が可能なシステムである.複数対複数の地点間による経路探索や,地図上に探索結果等っを図化することができ,不通過区間を設定することで現在の交通ネットワークだけでなく,過去や将来の交通ネットワークでも経路探索が可能である.\cite{NITAS}

\subsubsection{経路探索条件}
経路探索を行う時の詳細な設定について表\ref{conditions of search}に示す.
一般には所要時間最小ではなく一般化費用を最小とする条件で経路探索を行うが,所要時間を最小とすることで本研究のマーケットアクセスは企業の立地ポテンシャルを表し,産業目線での分析が可能になっていると言える.
\begin{table}[h!]
  \centering
  \renewcommand{\arraystretch}{1.2} % 行間を広げる
  \begin{tabularx}{\textwidth}{p{0.25\textwidth}   X}
  \toprule
  \textbf{設定名}     &\textbf{設定詳細}                                 \\
  \midrule
  探索条件            & 所要時間最小                                  \\
  設定速度            & 道路交通センサス2021年実施分より道路種類ごとに一定な平均旅行速度(混雑を考慮).\cite{kokudo2021} 過去の経路探索においても現況の速度と同じ速度を用いる.           \\
  交通モード            & 道路モード.有料道路と一般道路のみを使用し鉄道などの他の交通は使用しない.                                  \\
  ネットワーク設定          & 2021年〜2023年の道路ネットワークを不通過区間の設定により再現.ただし,有料道路以外は現況のネットワーク.                                  \\
  起終点設定            & 起点は各圏央道IC,終点は各放射道路のJCTから下り方面の一つ先のIC                               \\
  \bottomrule
  \end{tabularx}
  \caption{経路探索条件}
  \label{conditions of search}
  \end{table}
  



\subsubsection{乗降客数データ}
乗降客数データは国土数値情報の駅別乗降客数データを用いた.最寄り駅情報を用いて乗降客数を地価データに紐づけている.
\newpage
\section{データの作成}
\subsection{使用したデータ}
本研究では高速道路時系列データとNITASから得られた所要時間データを用いてICごとに接続本数データとマーケットアクセス指標を作成した.

\subsection{接続本数データ}
接続本数データとは,各時点において圏央道のICが圏央道を通じて接続している放射高速道路の本数のデータである.GIS形式の高速道路時系列データを用いて2001年〜2023年の圏央道の各ICが何本の放射高速道路と接続しているかを調査した.ICが供用開始前の場合は接続本数は0として扱った.接続本数データは地価データに最寄り圏央道IC情報を用いて紐付けた.表\ref{number of connections}に接続本数データの例を示す.

\begin{table}[h!]
  \centering
  \renewcommand{\arraystretch}{1.2} % 行間を広げる
  \begin{tabularx}{\textwidth}{p{0.15\textwidth} *{5}{p{0.075\textwidth}} p{0.075\textwidth}p{0.15\textwidth}}
  \toprule
  \textbf{IC名} & \textbf{2001} & \textbf{2002} & \textbf{2003} & \textbf{2004} & \textbf{2005} & \textbf{〜} & \textbf{供用開始年} \\
  \midrule
  あきる野 & 0 & 0 & 0 & 0 & 1 & 〜 & 2005 \\
  入間 & 1 & 1 & 1 & 1 & 1 & 〜 & 1996 \\
  つくば牛久 & 0 & 0 & 1 & 1 & 1 & 〜 & 2003 \\
  厚木PASIC & 0 & 0 & 0 & 0 & 0 & 〜 & 2020 \\
  \bottomrule
  \end{tabularx}
  \caption{接続本数データの例}
  \label{number of connections}
\end{table}



\subsection{マーケットアクセス指標}
本研究では織田澤・足立らを参考に,放射・環状型道路ネットワークを考慮した$n$年におけるIC$i$のマーケットアクセス指標を次のように定義する.
\[{MA}^{n}_{i} = \sum_{i \neq j}{t_{j}/c^{n}_{ij}}\]
ここで,$t_{j}$は放射高速道路$j$の上下線合計日交通量(台)であり,$c^{n}_{ij}$は$n$年におけるIC$i$と放射高速道路$j$の間の自動車による所要時間(分)である.放射高速道路の行き先の人口や雇用者密度などの経済規模ではなく放射構想道路の交通量を用いたのは,本研究が放射・環状型道路ネットワーク整備の効果の測定を目的としているためである.交通量を放射高速道路の価値と考えることで,放射高速道路の行き先の経済規模の影響を排除し放射・環状型道路ネットワークの接続性を示す指標としている.こうして求めたマーケットアクセスを地価データの最寄り圏央道IC情報を用いて地価観測点と紐づけた.

\section{データの記述統計}
 本節は分析に使用したデータについて記述統計を行う.2023年のマーケットアクセスの分布は図\ref{MA_2023_distribution}のようになっている.
 \begin{figure}[H]
  \centering
  \includegraphics[width=10cm]{figure/MA_distribution2.png}
  \caption{2023年のマーケットアクセスの分布}
  \label{MA_2023_distribution}
\end{figure}
図\ref{MA_2023_distribution}を見ると,マーケットアクセスは関越道周辺で大きく,ついで東北道や東名高速付近で大きくなっている.また,常磐道や東関東道周辺ではマーケットアクセスは小さいことが分かった.


\begin{figure}[H]
  \centering
  \includegraphics[width=7cm]{figure/ch_lap_dis_IC.png}
  \caption{地価の変化とICからの距離}
  \label{deruta_landprice disIC_m}
\end{figure}

\begin{figure}[H]
  \centering
  \includegraphics[width=7cm]{figure/ch_lap_2001_lap.png}
  \caption{地価の変化と2001年の地価}
  \label{deruta_landprice_H13_landprice}
\end{figure}






