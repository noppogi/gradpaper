\chapter{本研究の位置付け}

\section{既往研究の整理}
本節では,本研究に関係する既往研究を,市場へのアクセス性を表す指標としてマーケットアクセスを採用した研究,交通インフラネットワーク整備が周辺地域に及ぼす影響をマーケットアクセスを用いて分析した研究,交通インフラネットワーク整備が周辺地域に及ぼす影響をマーケットアクセス以外の指標を用いて分析した研究の3つに分類して紹介する.
第一に,市場へのアクセス性をを表す指標としてマーケットアクセスを採用した研究を紹介する.
Harris\cite{harrisc1954}は、地域$i$におけるマーケットアクセスを次のように定義した。
\[ MA = \sum_{i \neq j}{E_{j}/d_{ij}}\]
ここで,$E_{j}$は目的地の魅力を表し,人口や雇用者密度など経済規模に関する値が用いられる.$d_{ij}$は地域iとjの間の交通抵抗を表す.Hariisタイプのマーケットアクセス指標は多くの実証分析に用いられている.Redding and Vevables\cite{redding2004}は新経済地理学モデルに基づいてマーケットアクセスと賃金の関係式を導出し,マーケットアクセスの違いが一人当たりGDPの格差を生じさせることを世界101カ国を対象とした分析によって示した.Head and Mayer\cite{head2006}はEU諸国を対象にマーケットアクセスが賃金に及ぼす影響を明らかにした.また,Combes et al.\cite{combes2010}はマーケットアクセスを集積の対象として捉え,集積の便益の推計を行っている.しかし,これらの研究では交通抵抗として2地域間の直線距離を採用しているため,交通インフラのネットワーク性を十分に考慮することができていない.
 
続いて,交通インフラネットワーク整備が周辺地域に及ぼす影響をマーケットアクセスを用いて分析した研究を紹介する.
Donaldson and Hornbeck\cite{donaldson2016}は,鉄道ネットワーク整備がアメリカ経済に与えた影響について分析を行っている.具体的には,一般均衡型の都市間交易モデルに基づき地代とマーケットアクセスの関係性を導出し,1870年〜1890年における鉄道ネットワークの拡大が農業地価に及ぼした因果効果について群単位で推定している.マーケットアクセスの算出には,鉄道ネットワーク上のある群から別の群への最低輸送費用ルートを求め,その時の貿易コストを交通抵抗として用いている.推定の結果,マーケットアクセスの値が2倍変化すると,農業地価が51\%上昇することが明らかとなった.同様の枠組みを用いて,織田澤・足立ら\cite{otazawa2022}は日本の都市間高速道路ネットワークを対象に,地価とマーケットアクセスの関係式を導出して,高速道路ネットワークが地価に及ぼす因果効果を推定する.マーケットアクセスの算出には,地域ij間の自動車による交通所要時間を交通抵抗として用いている.推定の結果,マーケットアクセスが高ければ地価も高い水準にあるという因果効果を明らかにするとともに,「地価の低い地域においてマーケットアクセスが高い」という逆の因果性が存在する可能性を示した.

次に,交通インフラネットワーク整備が周辺地域に及ぼす影響をマーケットアクセス以外の指標を用いて分析した研究を紹介する.安藤・倉内\cite{kurauchi2020}は,岐阜県道路ネットワークを対象に交通容量とリンク長を考慮した固有ベクトル中心性指標を用いて道路ネットワークの評価を行った.固有ベクトル中心性指標とは,ネットワーク上で各ノードがどれほど中心的であるかを表す度合いで,需要データを必要とせず確率に依存することなく,ネットワークそのものが持つ形状の観点から評価を行うNetwork Topology指標の一種である.1985年〜2024年までの道路ネットワークの評価に加えて,交通容量増設施策の影響の検証を行った.春・内田\cite{haru2011}は,地域間の連結性を示す指標として到達率をグラフ理論を用いて定義し,中国西部地域を対象に道路ネットワーク整備が地域産業へ与える影響について分析を行った.分析の結果,到達率が高い地域ほど大きな産業成長をあげていることを明らかにした.

\section{本研究の位置付け}
% 本研究は,放射・環状道路ネットワークの整備効果がどのように発現するかを明らかにすることを目的としている.本研究では,首都圏郊外の圏央道をケーススタディとして取り上げ,圏央道整備によってアクセシビリティがどのように変化するかと,アクセシビリティの変化がその周辺地域に与えた影響について地価に着目して分析を行う.
% 本研究の新規性は,アクセシビリティの向上という道路ネットワークの機能に着目した分析を行った点,時間と共に一定のトレンドを持っているが,そのトレンドの度合いが個体ごとに異なる未観測因子である個別トレンドを考慮した分析を行った点にある.
本研究では,放射・環状型道路ネットワークの整備効果の発現特性を明らかにすることを目的としている.ケーススタディとして首都圏郊外の圏央道を取り上げ,圏央道整備によるアクセシビリティの変化が周辺地域に与えた影響を,放射高速道路接続本数やマーケットアクセスが地価に与える影響を推定することで明らかにする.分析では,2001年から2023年の23時点のパネルデータを用いて固定効果推定を行う.交互作用分析では,マーケットアクセスが地価に与える効果と地価観測点の特徴の交互作用についての分析を行い,マーケットアクセスが地価に与える影響に地価観測点の特徴による異質性が存在するかを明らかにする.

本研究の新規性は,放射・環状道路ネットワークのアクセシビリティの指標として放射高速道路接続本数と放射・環状道路ネットワークを考慮したマーケットアクセスを用いた点と,交互作用分析によって環状高速道路の整備効果に交互作用をもたらす要因を分析した点である.環状高速道路には放射高速道路同士を接続して放射・環状道路ネットワークを形成する役割があることから,圏央道を通じて接続する放射高速道路の本数でアクセシビリティを表現できると考えた.マーケットアクセスの算出には,分子に各放射高速道路の交通量,分母にNITASによる経路探索から得られた所要時間を用いた.経路探索の結果を用いているため,本研究のマーケットアクセスは,放射高速道路接続本数と比べて,ネットワークを考慮したアクセシビリティの指標であると考えた.交互作用分析では,ICまでの距離や最寄り駅の規模などの観点から交互作用の有無を分析した.

