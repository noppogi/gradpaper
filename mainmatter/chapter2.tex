\chapter{本研究の位置付け}

\section{既往研究の整理}
 一般にマーケットアクセス指標とは目的地へのアクセスのしやすさを表す指標である.Harris\cite{harrisc1954}は、地域iにおけるマーケットアクセスを次のように定義した。
\[ MA = \sum_{i \neq j}{E_{j}/d_{ij}}\]

ここで,$E_{j}$は目的地の魅力を表し,人口や雇用者密度など経済規模に関する値が用いられる.$d_{ij}$は地域iとjの間の交通抵抗を表す.Hariisタイプのマーケットアクセス指標は多くの実証分析に用いられている.Redding and Vevables\cite{redding2004}は新経済学地理学に基づいてマーケットアクセスと賃金の関係式を導出し,マーケットアクセスの違いが一人当たりGDPの格差を生じさせることを世界101カ国を対象とした分析によって示した.Head and Mayer\cite{head2006}はEU諸国を対象にマーケットアクセスが賃金に及ぼす影響を明らかにした.また,Combes et al.\cite{combes2010}はマーケットアクセスを集積の対象として捉え,集積の便益の推計を行っている.しかし,これらのの研究では交通抵抗として2地域間の直線距離を採用しているため,交通インフラのネットワーク性を十分に考慮することができていない.織田澤・足立ら\cite{otazawa2022}は交通抵抗にij間の自動車による交通所要時間を用いてマーケットアクセスと地価の関係式を導出し,マーケットアクセスが高ければ地価も高い水準にあるという因果効果を明らかにした.
 
\section{本研究の位置付け}
本研究は,放射・環状道路ネットワークの整備効果がどのように発現するかを明らかにすることを目的としている.本研究では,首都圏郊外の圏央道をケーススタディとして取り上げ,圏央道整備によってアクセシビリティがどのように変化するかと,アクセシビリティの変化がその周辺地域に与えた影響について地価に着目して分析を行う,
本研究の新規性は,アクセシビリティの向上という道路ネットワークの機能に着目した分析を行った点,時間と共に一定のトレンドを持っているが,そのトレンドの度合いが個体ごとに異なる未観測因子である個別トレンドを考慮した分析を行った点にある.