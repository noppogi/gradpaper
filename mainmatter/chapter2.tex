\chapter{既往研究の整理と本研究の位置付け}

\section{既往研究の整理}
 社会資本投資が地域経済に与える影響について、多くの実証研究が行われている.御園(2014)\cite{misono2014}は、日本の社会資本が地域別生産性に与える効果を再検証し、社会資本ストックが生産性向上に寄与することを示した. また、金(2013)\cite{kim2013}は、社会資本や公共投資の経済効果に関する実証研究を行い、公共投資が短期的な需要面への効果を持つことを指摘している. さらに、三井(2009)\cite{mitsui2009}は、動学パネルを用いて公的支出と地域経済成長の関係を検証し、社会資本が全体として成長率にプラスの効果を持つことを明らかにした. 一方、社会資本の種類によって効果が異なることも指摘されている.例えば、産業基盤や生活基盤関連の社会資本は有意にプラスの影響を与えるが、生産性向上に直接結びつかない社会資本の効果は限定的であるとされる\cite{mitsui2010}. また、社会資本と人的資本の関係に関する研究では、社会資本ストックが生産力効果を持つことが確認されている\cite{yamano2007}. これらの研究は、社会資本投資が地域経済に与える影響を理解する上で重要な知見を提供しているが、地域特性や投資の種類、規模によって効果が異なるため、さらなる分析が求められる.
 
\section{本研究の位置付け}
 本研究は、地域間の付加価値成長率と全要素生産性(TFP)を従属変数とし、社会資本投資が地域経済に与える影響を実証的に分析することを目的とする。特に、社会資本投資の経済効果を地域特性や投資規模の観点から評価し、その効果の異質性や波及メカニズムを明らかにすることに新規性を持つ。また、マクロ経済モデルを活用し、短期的な効果にとどまらず、長期的な持続可能性や地域間格差の是正に対する社会資本投資の役割を解明することを目指す。
さらに、本研究は、公共投資の効率性を向上させるための基盤的な知見を提供することを意図している。具体的には、社会資本投資がどのようにして地域の経済成長や生産性向上を支えるのかを詳細に分析し、効率的な資源配分や政策立案に資する実践的な指針を提示する。本研究の成果は、学術的な貢献に加え、地域経済政策における社会資本投資の最適なあり方を探る上での新たな示唆を与えるものである。
