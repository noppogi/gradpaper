\chapter{本研究の位置付け}

\section{既往研究の整理}
 一般にマーケットアクセス指標とは目的地へのアクセスのしやすさを表す指標である.Harris\cite{harrisc1954}は、地域iにおけるマーケットアクセスを次のように定義した。
\[ MA = \sum_{i \neq j}{E_{j}/d_{ij}}\]

ここで,$E_{j}$は目的地の魅力を表し,人口や雇用者密度など経済規模に関する値が用いられる.$d_{ij}$は地域iとjの間の交通抵抗を表す.Hariisタイプのマーケットアクセス指標は多くの実証分析に用いられている.Redding and Vevables\cite{redding2004}は新経済学地理学に基づいてマーケットアクセスと賃金の関係式を導出し,マーケットアクセスの違いが一人当たりGDPの格差を生じさせることを世界101カ国を対象とした分析によって示した.Head and Mayer\cite{head2006}はEU諸国を対象にマーケットアクセスが賃金に及ぼす影響を明らかにした.また,Combes et al.\cite{combes2010}はマーケットアクセスを集積の対象として捉え,集積の便益の推計を行っている.しかし,これらのの研究では交通抵抗として2地域間の直線距離を採用しているため,交通インフラのネットワーク性を十分に考慮することができていない.織田澤・足立ら\cite{otazawa2022}は交通抵抗にij間の自動車による交通所要時間を用いてマーケットアクセスと地価の関係式を導出し,マーケットアクセスが高ければ地価も高い水準にあるという因果効果を明らかにした.
 
\section{本研究の位置付け}
 本研究は、地域間の付加価値成長率と全要素生産性(TFP)を従属変数とし、社会資本投資が地域経済に与える影響を実証的に分析することを目的とする。特に、社会資本投資の経済効果を地域特性や投資規模の観点から評価し、その効果の異質性や波及メカニズムを明らかにすることに新規性を持つ。また、マクロ経済モデルを活用し、短期的な効果にとどまらず、長期的な持続可能性や地域間格差の是正に対する社会資本投資の役割を解明することを目指す。
さらに、本研究は、公共投資の効率性を向上させるための基盤的な知見を提供することを意図している。具体的には、社会資本投資がどのようにして地域の経済成長や生産性向上を支えるのかを詳細に分析し、効率的な資源配分や政策立案に資する実践的な指針を提示する。本研究の成果は、学術的な貢献に加え、地域経済政策における社会資本投資の最適なあり方を探る上での新たな示唆を与えるものである。
