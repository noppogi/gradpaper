\chapter{分析結果と考察}
\section{基本分析}
\subsection{接続本数分析}
分析結果を表\ref{result_setuzoku}に示す.接続本数の回帰係数は0.005となった.これは地価観測点の最寄りの圏央道のICの放射高速道路接続本数が1本増えると地価は0.5\%増加することを意味する.

また,1本目接続効果分析の結果を表\ref{result_heterogeneity_firstsetuzoku}に示す.1本目接続ダミーの回帰係数は有意な結果とならず,2本目以降接続本数の回帰係数は0.008で有意な結果となった.これより接続本数が0本から1本になることは地価に影響を与えるとは言えない.そして,2本目以降が接続すると地価を押し上げる効果があることが明らかになった.これらのことより,整備効果が発現するためには複数の放射高速道路に接続してネットワークを形成する必要があると考えられる.
\subsection{マーケットアクセス分析}
分析結果を表\ref{result_MA}に示す.
マーケットアクセスの回帰係数は0.139となった.これは地価観測点の最寄りの圏央道のICのマーケットアクセスが10\%増えると地価は1.39\%増えることを意味する.
\begin{table}[H]
  \centering
  \begin{minipage}[t]{0.45\columnwidth}
  \begin{center}
  \caption{分析結果 接続本数分析 \\ ~ }
  \label{result_setuzoku}
  \scalebox{0.80}{
  \def\sym#1{\ifmmode^{#1}\else\(^{#1}\)\fi}
  \begin{tabular}{l*{4}{c}}
  \hline\hline
                 &\multicolumn{1}{c}{}&\multicolumn{1}{c}{}&\multicolumn{1}{c}{Two-way}\\
  
  \hline
  (Intercept)    &       &      &       12.491\sym{***}&       \\
                 &       &      &        (0.006)       &        \\
  [1em]
  接続本数        &       &      &      0.005\sym{**} &          \\
                 &      &       &     (0.002)       &           \\
  [1em]
  \hline
  \(R^{2}\)      &      &       &        0.985        &          \\
  \(Adj.R^{2}\)  &      &       &       0.984         &         \\
  \(Num.obs.\)   &      &       &    20470            &          \\
  \(RMSE\)       &      &       &    0.084            &          \\
  \(N Clusters\) &      &       &    890              &          \\
  \hline\hline
  \multicolumn{5}{l}{\footnotesize Standard errors in parentheses}\\
  \multicolumn{5}{l}{\footnotesize \sym{*} \(p<0.05\), \sym{**} \(p<0.01\), \sym{***} \(p<0.001\)}\\
  \end{tabular}
  }
  \end{center}
  \end{minipage}
  \hfill
  \begin{minipage}[t]{0.45\columnwidth}
  \begin{center}
  \caption{分析結果 マーケットアクセス分析 \\~}
  \label{result_MA}
  \scalebox{0.80}{
  \def\sym#1{\ifmmode^{#1}\else\(^{#1}\)\fi}
  \begin{tabular}{l*{4}{c}}
    \hline\hline
    &\multicolumn{1}{c}{}&\multicolumn{1}{c}{}&\multicolumn{1}{c}{Two-way}\\

\hline
(Intercept)    &       &      &       11.236\sym{***}&       \\
               &       &      &        (0.291)       &        \\
[1em]
 MA            &       &      &      0.139\sym{***} &          \\
               &      &       &     (0.0321)       &           \\
[1em]
\hline
\(R^{2}\)      &      &       &        0.985        &          \\
\(Adj.R^{2}\)  &      &       &       0.985         &         \\
\(Num.obs.\)   &      &       &    20470            &          \\
\(RMSE\)       &      &       &    0.084            &          \\
\(N Clusters\) &      &       &    890              &          \\
\hline\hline
\multicolumn{5}{l}{\footnotesize Standard errors in parentheses}\\
\multicolumn{5}{l}{\footnotesize \sym{*} \(p<0.05\), \sym{**} \(p<0.01\), \sym{***} \(p<0.001\)}\\
\end{tabular}
}
  \end{center}
  \end{minipage}
  \end{table}
  


\begin{table}[H]
  \centering
  \begin{minipage}[t]{0.45\columnwidth}
  \begin{center}
  \caption{分析結果 1本目接続効果 \\~}
  \label{result_heterogeneity_firstsetuzoku}
  \scalebox{0.80}{
  \def\sym#1{\ifmmode^{#1}\else\(^{#1}\)\fi}
  \begin{tabular}{l*{4}{c}}
    \hline\hline
    &\multicolumn{1}{c}{}&\multicolumn{1}{c}{}&\multicolumn{1}{c}{Two-way}\\

\hline
(Intercept)    &       &      &       12.492\sym{***}&       \\
               &       &      &        (0.305)       &        \\
[1em]
 1本目接続ダミー            &      &       &      -0.003\sym{} &          \\
               &      &       &     (0.005)         &           \\
[1em]
2本目以降接続本数     &      &      &      0.008\sym{**} &          \\
               &      &       &     (0.003)         &           \\
  [1em]
\hline
\(R^{2}\)      &      &       &        0.985        &          \\
\(Adj.R^{2}\)  &      &       &       0.984         &         \\
\(Num.obs.\)   &      &       &    20470            &          \\
\(RMSE\)       &      &       &    0.084            &          \\
\(N Clusters\) &      &       &    890              &          \\
\hline\hline
\multicolumn{5}{l}{\footnotesize Clustered robust standard errors in parentheses}\\
\multicolumn{5}{l}{\footnotesize \sym{*} \(p<0.05\), \sym{**} \(p<0.01\), \sym{***} \(p<0.001\)}\\
\end{tabular}
}
  \end{center}
  \end{minipage}
  \end{table}
  


\section{交互作用分析}


\subsection{圏央道IC距離交互作用分析}
分析結果を表\ref{result_heterogeneity_disIC}に示す.
マーケットアクセスの回帰係数は0.161,マーケットアクセスと圏央道ICまでの距離の交差項の回帰係数は-0.046であるが有意な結果とはならなかった.この結果から,マーケットアクセスが地価に与える効果には圏央道ICまでの距離による異質性が存在するとはいえない.

このことより,マーケットアクセスが地価に与える効果は,マーケットアクセスが良いICを利用可能かどうかが重要であり,ICまでの距離によらずICが存在する地域全体に発現すると考えられる.
\subsection{2001年地価交互作用分析}
分析結果を表\ref{result_heterogeneity_Lp2001}に示す.
マーケットアクセスの回帰係数は0.462,マーケットアクセスと2001年の地価の交差項の回帰係数は-0.345となった.この結果からマーケットアクセスが地価に与える効果には基準時点の地価の高さによる異質性が存在すると言える.2001年の地価が25万円を超える地点と25万円以下の地点のどちらにおいても,マーケットアクセスは地価を押し上げる効果を持つが,2001年の地価が25万円を超える地点で効果が大きいことが明らかになった.

このことより,マーケットアクセスが地価を押し上げる効果は,元々地価が高い地点において地価が低い地点よりも大きく発現すると考えられる.
\begin{table}[H]
  \centering
  \begin{minipage}[t]{0.45\columnwidth}
  \begin{center}
  \caption{分析結果 IC距離 \\ ~ }
  \label{result_heterogeneity_disIC}
  \scalebox{0.80}{
  \def\sym#1{\ifmmode^{#1}\else\(^{#1}\)\fi}
  \begin{tabular}{l*{4}{c}}
  \hline\hline
                 &\multicolumn{1}{c}{}&\multicolumn{1}{c}{}&\multicolumn{1}{c}{Two-way}\\
  
  \hline
  (Intercept)    &       &      &       11.030\sym{***}&       \\
                 &       &      &        (0.314)       &        \\
  [1em]
  lnMA             &       &      &      0.161\sym{***} &          \\
                 &       &      &     (0.035)         &           \\
  [1em]
  \(lnMA*D^{IC}\)      &       &      &      -0.046\sym{}   &          \\
                 &       &      &     (0.027)         &           \\
  [1em]
  \hline
  \(R^{2}\)      &      &       &        0.985        &          \\
  \(Adj.R^{2}\)  &      &       &       0.985         &         \\
  \(Num.obs.\)   &      &       &    20470            &          \\
  \(RMSE\)       &      &       &    0.084            &          \\
  \(N Clusters\) &      &       &    890              &          \\
  \hline\hline
  \multicolumn{5}{l}{\footnotesize Clustered robust standard errors in parentheses}\\
  \multicolumn{5}{l}{\footnotesize \sym{*} \(p<0.05\), \sym{**} \(p<0.01\), \sym{***} \(p<0.001\)}\\
  \end{tabular}
  }
  \end{center}
  \end{minipage}
  \hfill
  \begin{minipage}[t]{0.45\columnwidth}
  \begin{center}
  \caption{分析結果 2001年地価 \\~}
  \label{result_heterogeneity_Lp2001}
  \scalebox{0.80}{
  \def\sym#1{\ifmmode^{#1}\else\(^{#1}\)\fi}
  \begin{tabular}{l*{4}{c}}
    \hline\hline
    &\multicolumn{1}{c}{}&\multicolumn{1}{c}{}&\multicolumn{1}{c}{Two-way}\\

\hline
(Intercept)    &       &      &       11.437\sym{***}&       \\
               &       &      &        (0.286)       &        \\
[1em]
 lnMA            &      &       &      0.462\sym{***} &          \\
               &      &       &     (0.065)         &           \\
[1em]
\(lnMA*D^{low}\)     &      &      &      -0.345\sym{***} &          \\
               &      &       &     (0.059)         &           \\
[1em]
\hline
\(R^{2}\)      &      &       &        0.986        &          \\
\(Adj.R^{2}\)  &      &       &       0.985         &         \\
\(Num.obs.\)   &      &       &    20470            &          \\
\(RMSE\)       &      &       &    0.082            &          \\
\(N Clusters\) &      &       &    890              &          \\
\hline\hline
\multicolumn{5}{l}{\footnotesize Clustered robust standard errors in parentheses}\\
\multicolumn{5}{l}{\footnotesize \sym{*} \(p<0.05\), \sym{**} \(p<0.01\), \sym{***} \(p<0.001\)}\\
\end{tabular}
}
  \end{center}
  \end{minipage}
  \end{table}
  


% \subsection{拠点駅交互作用分析}
% 分析結果を表\ref{result_heterogeneity_basestation}に示す.
% マーケットアクセスの回帰係数は0.099,マーケットアクセスと最寄り駅の交差項の回帰係数は0.117となった.この結果からマーケットアクセスが地価に対する影響には拠点駅の有無による異質性が存在し,最寄りの駅が拠点駅である地点においてそうでない地点よりも影響が大きいことが明らかになった.しかし,最寄り駅が拠点駅でない地点においてもマーケットアクセスの地価を上昇させる効果は存在する.

% このことより,マーケットアクセスの地価に対する影響は,
\subsection{乗降客数交互作用分析}
分析結果を表\ref{result_heterogeneity_passengers}に示す.
マーケットアクセスの回帰係数は0.237,マーケットアクセスと乗降客数の交差項の回帰係数は-0.200となった.この結果からマーケットアクセスが地価に与える影響は最寄り駅の乗降客数による異質性が存在すると言える.また,最寄りの乗降客数が5万人未満の地点と5万人を超える地点のどちらにおいても,マーケットアクセスは地価を押し上げる効果を持つが,乗降客数が5万人未満の地点の方が効果が小さいことが明らかになった.

このことより,マーケットアクセスが地価を押し上げる効果は,最寄り駅の規模の大きい地点において規模が小さい地点よりも大きく発現すると考えられる.

\subsection{最寄り駅距離交互作用分析}
分析結果を表\ref{result_heterogeneity_disstation}に示す.
マーケットアクセスの回帰係数は-0.047であるが有意な結果とはならず,マーケットアクセスと最寄り駅距離の回帰係数は0.303となった.マーケットアクセスと最寄り駅距離の交差項の回帰係数が有意な結果となったので,マーケットアクセスの地価に与える効果には最寄り駅までの距離による異質性が存在すると言える.しかし,マーケットアクセスの回帰係数が有意ではないので,最寄り駅との距離が2.5km以下の地点と2.5kmを超える地点の間の効果の違いがあることは明らかになったが,マーケットアクセスが地価を押し上げている効果があるとはいえない.

このことより,マーケットアクセスが地価に与える効果は,鉄道へのアクセスの違いによって異なると考えられる.
\begin{table}[H]
  \centering
  \begin{minipage}[t]{0.45\columnwidth}
  \begin{center}
    \caption{分析結果 乗降客数 \\ ~ }
    \label{result_heterogeneity_passengers}
    \scalebox{0.80}{
    \def\sym#1{\ifmmode^{#1}\else\(^{#1}\)\fi}
    \begin{tabular}{l*{4}{c}}
    \hline\hline
                   &\multicolumn{1}{c}{}&\multicolumn{1}{c}{}&\multicolumn{1}{c}{Two-way}\\
    
    \hline
    (Intercept)    &       &      &       12.169\sym{***}&       \\
                   &       &      &        (0.298)       &        \\
    [1em]
    lnMA             &       &      &      0.237\sym{***} &          \\
                   &       &      &     (0.035)         &           \\
    [1em]
    \(lnMA*D^{pas}\)      &       &      &      -0.200\sym{***}   &          \\
                   &       &      &     (0.027)         &           \\
    [1em]
    \hline
    \(R^{2}\)      &      &       &        0.986        &          \\
    \(Adj.R^{2}\)  &      &       &       0.985         &         \\
    \(Num.obs.\)   &      &       &    20447            &          \\
    \(RMSE\)       &      &       &    0.082            &          \\
    \(N Clusters\) &      &       &    889              &          \\
  \hline\hline
  \multicolumn{5}{l}{\footnotesize Clustered robust standard errors in parentheses}\\
  \multicolumn{5}{l}{\footnotesize \sym{*} \(p<0.05\), \sym{**} \(p<0.01\), \sym{***} \(p<0.001\)}\\
  \end{tabular}
  }
  \end{center}
  \end{minipage}
  \hfill
  \begin{minipage}[t]{0.45\columnwidth}
  \begin{center}
  \caption{分析結果 最寄り駅距離 \\~}
  \label{result_heterogeneity_disstation}
  \scalebox{0.80}{
  \def\sym#1{\ifmmode^{#1}\else\(^{#1}\)\fi}
  \begin{tabular}{l*{4}{c}}
    \hline\hline
    &\multicolumn{1}{c}{}&\multicolumn{1}{c}{}&\multicolumn{1}{c}{Two-way}\\

\hline
(Intercept)    &       &      &       10.159\sym{***}&       \\
               &       &      &        (0.305)       &        \\
[1em]
 lnMA            &      &       &      -0.047\sym{} &          \\
               &      &       &     (0.032)         &           \\
[1em]
\(lnMA*D^{sta}\)     &      &      &      0.303\sym{***} &          \\
               &      &       &     (0.027)         &           \\
[1em]
\hline
\(R^{2}\)      &      &       &        0.986        &          \\
\(Adj.R^{2}\)  &      &       &       0.986         &         \\
\(Num.obs.\)   &      &       &    20470            &          \\
\(RMSE\)       &      &       &    0.081            &          \\
\(N Clusters\) &      &       &    890              &          \\
\hline\hline
\multicolumn{5}{l}{\footnotesize Clustered robust standard errors in parentheses}\\
\multicolumn{5}{l}{\footnotesize \sym{*} \(p<0.05\), \sym{**} \(p<0.01\), \sym{***} \(p<0.001\)}\\
\end{tabular}
}
  \end{center}
  \end{minipage}
  \end{table}
  

  % \hline
  % (Intercept)    &       &      &       11.602\sym{***}&       \\
  %                &       &      &        (0.302)       &        \\
  % [1em]
  % lnMA             &       &      &      0.099\sym{**} &          \\
  %                &       &      &     (0.034)         &           \\
  % [1em]
  % \(lnMA*D^{base}\)      &       &      &      0.117\sym{**}   &          \\
  %                &       &      &     (0.036)         &           \\
  % [1em]
  % \hline
  % \(R^{2}\)      &      &       &        0.985        &          \\
  % \(Adj.R^{2}\)  &      &       &       0.985         &         \\
  % \(Num.obs.\)   &      &       &    20470            &          \\
  % \(RMSE\)       &      &       &    0.083            &          \\
  % \(N Clusters\) &      &       &    890              &          \\
\subsection{最寄りIC距離・最寄り駅距離交互作用分析}
分析結果を表\ref{result_heterogeneity_disstation_disIC}に示す.
マーケットアクセスの回帰係数は-0.043であるが有意な結果とはならず,マーケットアクセスと最寄り駅距離の交差項の回帰係数は0.302,マーケットアクセスとICまでの距離の交差項は-0.006であるが有意な結果とはならなかった.この結果からマーケットアクセスが地価に与える影響にはICまでの距離による異質性は存在するといえないが,最寄り駅までの距離による異質性が存在することが明らかになった.しかし,最寄り駅距離交互作用分析と同じく,マーケットアクセスの回帰係数は有意ではないためマーケットアクセスが地価を押し上げている効果があるとはいえない.
\begin{table}[H]
  \centering
  \begin{minipage}[t]{0.45\columnwidth}
  \begin{center}
  \caption{分析結果 乗降客数 \\ ~ }
  \label{result_heterogeneity_passengers}
  \scalebox{0.80}{
  \def\sym#1{\ifmmode^{#1}\else\(^{#1}\)\fi}
  \begin{tabular}{l*{4}{c}}
  \hline\hline
                 &\multicolumn{1}{c}{}&\multicolumn{1}{c}{}&\multicolumn{1}{c}{Two-way}\\
  
  \hline
  (Intercept)    &       &      &       12.169\sym{***}&       \\
                 &       &      &        (0.298)       &        \\
  [1em]
  lnMA             &       &      &      0.237\sym{***} &          \\
                 &       &      &     (0.035)         &           \\
  [1em]
  \(lnMA*D^{sta}\)      &       &      &      -0.200\sym{***}   &          \\
                 &       &      &     (0.027)         &           \\
  [1em]
  
  \hline
  \(R^{2}\)      &      &       &        0.986        &          \\
  \(Adj.R^{2}\)  &      &       &       0.985         &         \\
  \(Num.obs.\)   &      &       &    20447            &          \\
  \(RMSE\)       &      &       &    0.082            &          \\
  \(N Clusters\) &      &       &    889              &          \\
  \hline\hline
  \multicolumn{5}{l}{\footnotesize Clustered robust standard errors in parentheses}\\
  \multicolumn{5}{l}{\footnotesize \sym{*} \(p<0.05\), \sym{**} \(p<0.01\), \sym{***} \(p<0.001\)}\\
  \end{tabular}
  }
  \end{center}
  \end{minipage}
  \hfill
  \begin{minipage}[t]{0.45\columnwidth}
  \begin{center}
  \caption{分析結果 IC距離と最寄り駅距離 \\~}
  \label{result_heterogeneity_disstation_disIC}
  \scalebox{0.80}{
  \def\sym#1{\ifmmode^{#1}\else\(^{#1}\)\fi}
  \begin{tabular}{l*{4}{c}}
    \hline\hline
    &\multicolumn{1}{c}{}&\multicolumn{1}{c}{}&\multicolumn{1}{c}{Two-way}\\

\hline
(Intercept)    &       &      &       10.134\sym{***}&       \\
               &       &      &        (0.305)       &        \\
[1em]
 lnMA            &      &       &      -0.043\sym{} &          \\
               &      &       &     (0.034)         &           \\
[1em]
\(lnMA*D^{sta}\)     &      &      &      0.302\sym{***} &          \\
               &      &       &     (0.027)         &           \\
[1em]
\(lnMA*D^{IC}\)      &       &      &      -0.006\sym{}   &          \\
                 &       &      &     (0.025)         &           \\
  [1em]
\hline
\(R^{2}\)      &      &       &        0.986        &          \\
\(Adj.R^{2}\)  &      &       &       0.986         &         \\
\(Num.obs.\)   &      &       &    20470            &          \\
\(RMSE\)       &      &       &    0.081            &          \\
\(N Clusters\) &      &       &    890              &          \\
\hline\hline
\multicolumn{5}{l}{\footnotesize Clustered robust standard errors in parentheses}\\
\multicolumn{5}{l}{\footnotesize \sym{*} \(p<0.05\), \sym{**} \(p<0.01\), \sym{***} \(p<0.001\)}\\
\end{tabular}
}
  \end{center}
  \end{minipage}
  \end{table}
  


