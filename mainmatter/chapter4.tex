\chapter{分析}

\section{固定効果推定}
固定効果モデルはパネルデータを用いた因果推論に広く用いられているモデルである.固定効果モデルは,同一個体の時間的な変動に着目することで個体が持つ個体間で異なる時間不変の未観測因子を個別固定効果として調整することを得意とする.個別固定効果項を調整することで欠落変数バイアスが生じることを防ぐことができる.固定効果モデルは式\ref{4.1}で示される回帰モデルである.
\begin{equation}
  Y_{it} = \alpha_{i} + \beta_{1}X_{1it} + \beta_{2}X_{2it} + \cdots  + \beta_{k}X_{kit} + \varepsilon_{it}
  \label{4.1}
\end{equation}
それぞれの$i=1,2,...,N$および$t=1,2,...,T$について,$Y_{it}$は被説明変数であり,($X_{1it},X_{2it},...,X_{kit}$)は説明変数を示す.
また,$\alpha_{i}$は個体$i$の時間不変の固定効果,すなわち個別固定効果を,$\varepsilon_{it}$は誤差項を示す.
個別固定効果モデルでは欠落変数バイアスをもたらす効果は,時間を通じて一定であると仮定する.

式\ref{4.1}は個体のみの固定効果を考慮したモデルであるが,個体に加えて時点の固定効果を考慮した固定効果モデルも存在する.時点固定効果とは個体間では等しいが時間と共に変化する未観測因子のことを指し,個体固定効果と時点固定効果を同時に調整する固定効果モデルは双方向固定効果モデルと呼ばれる.双方向固定効果モデルは式\ref{4.2}で示される回帰モデルである.
\begin{equation}
  Y_{it} = \alpha_{i} + \pi_{t} + \beta_{1}X_{1it} + \beta_{2}X_{2it} + \cdots  + \beta_{k}X_{kit} + \varepsilon_{it}
  \label{4.2}
\end{equation}
それぞれの$i=1,2,...,N$および$t=1,2,...,T$について,$Y_{it}$は被説明変数であり,($X_{1it},X_{2it},...,X_{kit}$)は説明変数を示す.
また,$\alpha_{i}$は個体$i$の個別固定効果を,$\pi_{t}$は時間$t$における時点固定効果を,$\varepsilon_{it}$は誤差項を示す.

固定効果モデルの推定の代表的な方法としてLSDV推定がある.双方向固定効果モデルの推定において,LSDV推定では個体を表すダミー変数を$N$個と時点を表すダミー変数を$T$個を推定の式に含め,それを最小二乗推定する.
\begin{equation}
  \begin{aligned}
    Y_{it} = \beta_{1}X_{1it} + \cdots  + \beta_{k}X_{kit} \\
    &+ \gamma_{1}I_{1i} + \cdots + \gamma_{N}I_{Ni} + \delta_{1}D_{1t} + \cdots + \delta_{T}D_{Tt} + \varepsilon_{it}
  \end{aligned}
  \label{4.3} 
\end{equation}
式\ref{4.3}において,$I_{1i},...,I_{Ni}$は個体$i$ごとのダミー変数,$D_{1t},...,D_{Tt}$は時点$t$ごとのダミー変数を表す.
\begin{equation}
  D_{ji} =
  \begin{cases}
    1 & \text{if } j = i \\
    0 & \text{otherwise}
  \end{cases}
\end{equation}
\begin{equation}
  I_{lt} =
  \begin{cases}
    1 & \text{if } l = t \\
    0 & \text{otherwise}
  \end{cases}
\end{equation}

実際の計算では多重共線性を回避するため,ダミー変数をのうちどれか一つを削除する必要があり,通常は統計ソフトがどれかを自動的に削除する.また,回帰分析では残差が独立に同一の分布に従うという仮定のもとで標準誤差を計算する必要があるが,パネルデータ分析の場合はこの仮定が成り立たない可能性が非常に高く,一般には個体(クラスター)内の相関を許容するクラスター・ロバスト標準誤差の使用が強く推奨される.

\section{分析手法}
本研究では双方向固定効果モデルを用いて,基本分析として接続本数分析とマーケットアクセス分析を行い,個別トレンド分析として距離個別トレンド分析と初期地価個別トレンド分析を行った.推定にはLSDV推定を用いクラスター・ロバスト標準誤差を使用した.
\subsection{基本分析}
\subsubsection{接続本数分析}
接続本数分析での推定には式\ref{4.4}の双方向固定効果モデルを用いた.
\begin{eqnarray}
    \log{Y_{it}}= &&\beta_{C}C_{it} + \gamma_{1}I_{1i} + \cdots + \gamma_{N}I_{Ni} \nonumber \\
    && + \delta_{2001}D_{2001t} + \cdots + \delta_{2023}D_{2023t} + \varepsilon_{it}
  \label{4.4}
\end{eqnarray}
ここで,$Y_{it}$は地価観測点$i$の時点$t$における地価,$C_{it}$は地価観測点$i$の時点$t$における放射道路の接続数,$I_{1i},...,I_{Ni}$は各個体ダミー,$D_{2001t},...,D_{2023t}$は2001年から2023年の各年次ダミー,$\varepsilon_{it}$は誤差項である.TWFE推定量$\beta_{C}$は地価観測点において放射道路の接続数が1本増えた時に地価が何%変化するかを表す.

\subsubsection{マーケットアクセス分析}
マーケットアクセス分析の推定では式\ref{4.5}の双方向固定効果モデルを用いた.

\begin{eqnarray}
  \log{Y_{it}}= &&\beta_{MA}\log{{MA}_{it}}+ \gamma_{1}I_{1i} + \cdots + \gamma_{N}I_{Ni}  \nonumber \\
    && + \delta_{2001}D_{2001t} + \cdots + \delta_{2023}D_{2023t} + \varepsilon_{it}
    \label{4.5}
\end{eqnarray}  
ここで,$Y_{it}$は地価観測点$i$の時点$t$における地価,$MA_{it}$は地価観測点$i$の時点$t$におけるマーケットアクセス,$I_{1i},...,I_{Ni}$は各個体ダミー,$D_{2001t},...,D_{2023t}$は2001年から2023年の各年次ダミー,$\varepsilon_{it}$は誤差項である.TWFE推定量$\beta_{MA}$は地観測点においてマーケットアクセスが1%増加したときに地価が何%変化するかを表す.
\subsection{個別トレンド分析}
個別トレンド分析では,双方向固定効果モデルに個別トレンド項を追加したモデルを用いた.基本分析では時間を通じて一定であるが個体ごとに異なる未観測因子である個別固定効果や,個体間では等しいが時間と共に変化する未観測因子である時点固定効果を調整して圏央道の整備効果を推定していたが,そこに加えて個別トレンド分析では時間と共に一定のトレンドを持っているが,そのトレンドの度合いが個体ごとに異なる未観測因子である個別トレンドを調整して圏央道の整備効果を推定することを目的としている.

式\ref{4.6}は式\ref{4.2}の双方向固定効果モデルに個別トレンド項$\lambda_{i}t$を加えたモデルである.
\begin{eqnarray}
  Y_{it}= \alpha_{i} + \pi_{t} +\lambda_{i}t + \beta_{1}X_{it} + \cdots  + \beta_{k}X_{kit} + \varepsilon_{it}
\label{4.6} 
\end{eqnarray}
ここで,$\lambda_{i}$は個体ごとに異なる係数,$t$はトレンド変数である.個別トレンドを含んだ双方向固定効果モデルを推定する際は,ダミー変数を加えLSDV推定を用いて式\ref{4.7}を最小二乗推定する.
\begin{eqnarray}
  Y_{it} = &&\beta_{1}X_{1it} + \cdots  + \beta_{k}X_{kit} + \lambda_{1}I_{it}t+ \cdots  + \lambda_{N}I_{it}t \nonumber \\
    &&+ \gamma_{1}I_{1i} + \cdots + \gamma_{N}I_{Ni} + \delta_{1}D_{1t} + \cdots + \delta_{T}D_{Tt} + \varepsilon_{it}
  \label{4.7}
\end{eqnarray}
それぞれの$i=1,2,...,N$および$t=1,2,...,T$について,$Y_{it}$は被説明変数であり,($X_{1it},X_{2it},...,X_{kit}$)は説明変数を示す.
また,$I_{1i},...,I_{Ni}$は個体$i$ごとのダミー変数,$D_{1t},...,D_{Tt}$は時点$t$ごとのダミー変数,$\varepsilon_{it}$は誤差項を示す.実際の計算では双方向固定効果モデルと同じく多重共線性を回避するためダミー変数の中からどれか一つを削除する必要があり,通常は統計ソフトがどれかを自動的に削除する.
また,個別トレンド分析では被説明変数に地価,説明変数に各観測点のマーケットアクセス指標を用いる.

\subsubsection{距離個別トレンド分析}
距離個別トレンド分析ではICからの距離による個別トレンドをモデルに加えて圏央道の整備効果の推定を行う.基本分析では各観測点の最寄りのICが接続している放射道路の本数や,マーケットアクセスを用いて整備効果の推定を行ったが,各観測点からICまでの距離は考慮されていない.図\ref{deruta_landprice disIC_m}からは,ICからの距離の違いによって地価の変化のトレンドはほぼ変化していないと読み取れるが,僅かな個別トレンドも調整することで,より正確な圏央道の整備効果を推定できると考える.LSDV推定を行う際のICまでの距離による個別トレンドを調整した双方向固定効果モデルは式\ref{4.8}で表される.
\begin{eqnarray}
  \log{Y_{it}}= &&\beta_{MA}\log{{MA}_{it}}+ \gamma_{1}I_{1i} + \cdots + \gamma_{N}I_{Ni} + \lambda_{1}d_{i}I_{it}t+ \cdots  + \lambda_{N}d_{i}I_{it}t \nonumber \\
    && + \delta_{2001}D_{2001t} + \cdots + \delta_{2023}D_{2023t} + \varepsilon_{it}
  \label{4.8}
\end{eqnarray}
ここで,$Y_{it}$は地価観測点$i$の時点$t$における地価,$MA_{it}$は地価観測点$i$の時点$t$におけるマーケットアクセス,$d_{i}$は地価観測点$i$の最寄りICまでの距離,$I_{1i},...,I_{Ni}$は各個体ダミー,$D_{2001t},...,D_{2023t}$は2001年から2023年の各年次ダミー,$\varepsilon_{it}$は誤差項である.

\subsubsection{初期地価個別トレンド分析}
初期地価個別トレンド分析では2001年における各観測点の地価による個別トレンドをモデルに加え圏央道の推定を行う.図\ref{deruta_landprice_H13_landprice}からは,2001年時点の地価が高いと地価が大きく減るというトレンドを読み取ることができる.よって2001年時点での地価の大きさによって地価の変化のトレンドが異なることを考慮して整備効果を推定する必要があると考えられる.LSDV推定を行う際の2001年時程の地価による個別トレンドを調整した双方向固定効果モデルは式\ref{4.9}で表される.
\begin{eqnarray}
  \log{Y_{it}}= &&\beta_{MA}\log{{MA}_{it}}+ \gamma_{1}I_{1i} + \cdots + \gamma_{N}I_{Ni} + \lambda_{1}L_{i}I_{it}t+ \cdots  + \lambda_{N}L_{i}I_{it}t \nonumber \\
    && + \delta_{2001}D_{2001t} + \cdots + \delta_{2023}D_{2023t} + \varepsilon_{it}
  \label{4.9}
\end{eqnarray}
ここで,$Y_{it}$は地価観測点$i$の時点$t$における地価,$MA_{it}$は地価観測点$i$の時点$t$におけるマーケットアクセス,$L_{i}$は地価観測点$i$の最寄りICまでの距離,$I_{1i},...,I_{Ni}$は各個体ダミー,$D_{2001t},...,D_{2023t}$は2001年から2023年の各年次ダミー,$\varepsilon_{it}$は誤差項である