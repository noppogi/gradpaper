\chapter{分析}

\section{固定効果モデル}
固定効果モデル(Fixed Effect Model:FE)はパネルデータを用いた因果推論に広く用いられているモデルである.固定効果モデルは,同一個体の時間的な変動に着目することで個体が持つ個体間で異なる時間不変の未観測因子を個別固定効果として調整することを得意とする.個別固定効果項を調整することで欠落変数バイアスが生じることを防ぐことができる.固定効果モデルは式\ref{4.1}で示される回帰モデルである.
\begin{equation}
  Y_{it} = \alpha_{i} + \beta_{1}X_{1it} + \beta_{2}X_{2it} + \cdots  + \beta_{k}X_{kit} + \varepsilon_{it}
  \label{4.1}
\end{equation}
それぞれの$i=1,2,...,N$および$t=1,2,...,T$について,$Y_{it}$は被説明変数であり,($X_{1it},X_{2it},...,X_{kit}$)は説明変数を示す.
また,$\alpha_{i}$は個体$i$の時間不変の固定効果,すなわち個別固定効果を,$\varepsilon_{it}$は誤差項を示す.
個別固定効果モデルでは欠落変数バイアスをもたらす効果は,時間を通じて一定であると仮定する.

式\ref{4.1}は個体の固定効果のみを考慮したモデルであるが,個体に加えて時点の固定効果を考慮した固定効果モデルも存在する.時点固定効果とは個体間では等しいが時間と共に変化する未観測因子のことを指し,個体固定効果と時点固定効果を同時に調整する固定効果モデルは双方向固定効果モデル(Two-Way Fixed Effects Model:TWFE)と呼ばれる.
双方向固定効果モデルは式\ref{4.2}で示される回帰モデルである.
\begin{equation}
  Y_{it} = \alpha_{i} + \pi_{t} + \beta_{1}X_{1it} + \beta_{2}X_{2it} + \cdots  + \beta_{k}X_{kit} + \varepsilon_{it}
  \label{4.2}
\end{equation}
それぞれの$i=1,2,...,N$および$t=1,2,...,T$について,$Y_{it}$は被説明変数であり,($X_{1it},X_{2it},...,X_{kit}$)は説明変数である.
また,$\alpha_{i}$は個体$i$の個別固定効果を,$\pi_{t}$は時間$t$における時点固定効果を,$\varepsilon_{it}$は誤差項を表す.

固定効果モデルの推定の代表的な方法としてLSDV推定がある.LSDV推定では個体ごとのダミー変数と時点ごとのダミー変数を,それぞれ$N$個,$T$個推定の式に含め,それを最小二乗推定する.
\begin{equation}
  \begin{aligned}
    Y_{it} = \beta_{1}X_{1it} + \cdots  + \beta_{k}X_{kit} \\
    &+ \gamma_{1}I1_{it} + \cdots + \gamma_{N}IN_{it} + \delta_{1}D1_{it} + \cdots + \delta_{T}DT_{it} + \varepsilon_{it}
  \end{aligned}
  \label{4.3} 
\end{equation}
式\ref{4.3}において,$I1_{it},...,IN_{it}$は個体ごとのダミー変数,$D1_{it},...,DT_{it}$は時点ごとのダミー変数である.
\begin{equation}
  DJ_{it} =
  \begin{cases}
    1 & \text{if } J = i \\
    0 & \text{otherwise}
  \end{cases}
\end{equation}
\begin{equation}
  IL_{it} =
  \begin{cases}
    1 & \text{if } L = t \\
    0 & \text{otherwise}
  \end{cases}
\end{equation}

実際の計算では多重共線性を回避するため,ダミー変数をのうちどれか一つを削除する必要があり,通常は統計ソフトがどれかを自動的に削除する.また,回帰分析では残差が独立に同一の分布に従うという仮定のもとで標準誤差を計算する必要があるが,パネルデータ分析の場合はこの仮定が成り立たない可能性が非常に高く,一般には個体(クラスター)内の相関を許容するクラスター・ロバスト標準誤差の使用が強く推奨される.

\section{分析手法}
本研究では双方向固定効果モデルを用いて,基本分析として接続本数分析とマーケットアクセス分析を行い,発展分析として交互作用分析を行った.推定にはLSDV推定を用いクラスター・ロバスト標準誤差を使用した.
\subsection{基本分析}
\subsubsection{接続本数分析}
接続本数分析での推定には式\ref{4.4}の双方向固定効果モデルを用いた.
\begin{eqnarray}
    \ln{Y_{it}}= \alpha_{i} + \pi_{t} + \beta_{C}C_{it} + \varepsilon_{it}
  \label{4.4}
\end{eqnarray}
ここで,$Y_{it}$は地価観測点$i$の時点$t$における地価,$\alpha_{i}$は個別固定効果,$\pi_{t}$は時間固定効果,$C_{it}$は地価観測点$i$の時点$t$における放射高速道路の接続本数,$\varepsilon_{it}$は誤差項である.

TWFE推定量$\beta_{C}$は地価観測点において放射高速道路の接続本数が1本増えた時に地価が何\%変化するかを表す.この分析では接続本数データを接続性の指標として扱っている.

また,接続本数の中でも1本目の放射高速道路が接続した時の効果に着目した分析も行った.1本目接続効果分析では式\ref{first_setuzoku}の双方向固定効果モデルを用いた.
\begin{eqnarray}
  \ln{Y_{it}}= \alpha_{i} + \pi_{t} + \beta_{first}D^{first}_{it} + \beta_{second}C^{second}_{it} + \varepsilon_{it}
\label{first_setuzoku}
\end{eqnarray}
ここで,$Y_{it}$は地価観測点$i$の時点$t$における地価,$\alpha_{i}$は個別固定効果,$\pi_{t}$は時間固定効果,$D^{first}_{it}$は1本目接続ダミー,$C^{second}_{it}$は地価観測点$i$の時点$t$における放射高速道路の2本目以降の接続本数,$\varepsilon_{it}$は誤差項である.

\subsubsection{マーケットアクセス分析}
マーケットアクセス分析の推定には式\ref{4.5}の双方向固定効果モデルを用いた.
\begin{eqnarray}
  \ln{Y_{it}}= \alpha_{i} + \pi_{t} + \beta_{MA}\ln{{MA}_{it}}+ \varepsilon_{it}
    \label{4.5}
\end{eqnarray}  
ここで,$Y_{it}$は地価観測点$i$の時点$t$における地価,$\alpha_{i}$は個別固定効果,$\pi_{t}$は時間固定効果,$MA_{it}$は地価観測点$i$の時点$t$におけるマーケットアクセス,$\varepsilon_{it}$は誤差項である.

TWFE推定量$\beta_{MA}$は地価観測点においてマーケットアクセスが1\%増加したときに地価が何\%変化するかを表す.この分析ではマーケットアクセスを放射・環状型ネットワークを考慮した接続性の指標として扱っている.




\subsection{交互作用分析}
基本分析では説明変数$X$と被説明変数$Y$の間の関係を線形と仮定して推定を行った.この仮定によって,推定された回帰係数はある一つの値に決定される.しかし,図\ref{interaction_term}のように$X$が$Y$に与える影響がその他の変数$Z$の影響を受ける場合を考えると,$X$が$Y$に与える影響は一定ではない可能性がある.$Z$によって$X$が$Y$に与える影響が変わる場合,「$X$が$Y$に与える影響と$Z$に交互作用がある」ということができ,この$Z$は調整変数と呼ばれる.交互作用分析では調整変数による交互作用を考慮した双方向固定効果モデルを用いる.調整変数を含んだ2変数の双方向固定効果モデルは式\ref{interaction}で示される回帰モデルである.
\begin{figure}[H]
  \centering
  \includegraphics[width=10cm]{figure/interaction_term.jpeg}
  \caption{交互作用}
  \label{interaction_term}
\end{figure}
\begin{equation}
  Y_{it} = \alpha_{i} + \pi_{t} + \beta_{1}X_{it} + \beta_{2}Z_{i}X_{it} + \varepsilon_{it}
  \label{interaction}
\end{equation}
ここで,$Z_{i}X_{it}$は交差項と呼ばれる.$Z$がダミー変数の場合を考えると式\ref{Z_dummy}のようになる.
\begin{equation}
  Y_{it} =
  \begin{cases}
    \alpha_{i} + \pi_{t} + \beta_{1}X_{it}  + \varepsilon_{it} & \text (Z_{i} = 0) \\
    \alpha_{i} + \pi_{t} + (\beta_{1}+\beta_{2})X_{it} + \varepsilon_{it} & \text (Z_{i} = 1)
  \end{cases}
  \label{Z_dummy}
\end{equation}
式\ref{Z_dummy}から$Z$の値によって回帰直線の「傾き」が変わることが分かる.$\beta_{1}$は$Z=0$のときの$X$が$Y$に与える影響を意味し,$\beta_{2}$は$Z=1$のときと$Z=0$のときとの$X$が$Y$に与える影響の差を意味する.

交互作用分析では,マーケットアクセス分析を基本の形としており,様々な観点から分析を行う.
\subsubsection{IC距離交互作用分析}
IC距離交互作用分析では,マーケットアクセスが地価に与える影響の,各地価観測点から最寄りの圏央道ICまでの距離による変化について分析を行う.IC距離交互作用分析での推定には式\ref{interaction_disIC}に示される双方向固定効果モデルを用いた.
\begin{equation}
  \ln{Y_{it}} = \alpha_{i} + \pi_{t} + \beta\ln{MA_{it}} + \beta^{IC}Z^{IC}_{i}\ln{MA_{it}} + \varepsilon_{it}
  \label{interaction_disIC}
\end{equation}
ここで,$Y_{it}$は地価観測点$i$の時点$t$における地価,$\alpha_{i}$は個別固定効果,$\pi_{t}$は時間固定効果,$MA_{it}$は地価観測点$i$の時点$t$におけるマーケットアクセス,$Z^{IC}_{i}$は地価観測点$i$から最寄りの圏央道ICまでの距離が5km以下のときに1,それ以外のときに0をとるダミー変数,$\varepsilon_{it}$は誤差項である.

この分析において$\beta$は最寄りの圏央道ICまでの距離が5kmを超えるときのマーケットアクセスが地価に与える効果,$\beta^{IC}$は最寄りの圏央道ICまでの距離が5km以下のときと,5kmを超えるときとのマーケットアクセスが地価に与える効果の差を表す.
\subsubsection{2001年地価交互作用分析}
2001年地価交互作用分析では,マーケットアクセスが地価に与える影響の,各地価観測点の2001年時点の地価による変化について分析を行う.2001年を基準時点とすることで,元々の地価の高さによる交互作用を分析する.2001年地価交互作用分析での推定には式\ref{interaction_2001LP}に示される双方向固定効果モデルを用いた.
\begin{equation}
  \ln{Y_{it}} = \alpha_{i} + \pi_{t} + \beta\ln{MA_{it}} + \beta^{low}Z^{low}_{i}\ln{MA_{it}} + \varepsilon_{it}
  \label{interaction_2001LP}
\end{equation}
ここで,$Y_{it}$は地価観測点$i$の時点$t$における地価,$\alpha_{i}$は個別固定効果,$\pi_{t}$は時間固定効果,$MA_{it}$は地価観測点$i$の時点$t$におけるマーケットアクセス,$Z^{low}_{i}$は地価観測点$i$の2001年の地価が25万円以下のときに1,それ以外のときに0をとるダミー変数,$\varepsilon_{it}$は誤差項である.

この分析において$\beta$は2001年の地価が25万円を超えるときのマーケットアクセスが地価に与える効果,$\beta^{low}$は2001年の地価が25万円以下のときと,25万円を超えるときとのマーケットアクセスが地価に与える効果の差を表す.
% \subsubsection{拠点駅交互作用分析}
% 拠点駅交互作用分析では,マーケットアクセスが地価に与える影響が,各地価観測点の最寄りの駅が拠点駅かどうかによって変化するかどうかについて分析を行う.拠点駅とは2路線以上と接続している駅を指す.拠点駅交互作用分析での推定には式\ref{interaction_basestation}に示される双方向固定効果モデルを用いた.
% \begin{equation}
%   \ln{Y_{it}} = \alpha_{i} + \pi_{t} + \beta\ln{MA_{it}} + \beta^{base}Z^{base}_{i}\ln{MA_{it}} + \varepsilon_{it}
%   \label{interaction_basestation}
% \end{equation}
% ここで,$Y_{it}$は地価観測点$i$の時点$t$における地価,$MA_{it}$は地価観測点$i$の時点$t$におけるマーケットアクセス,$Z^{base}_{i}$は地価観測点$i$の最寄り駅が拠点駅の時に1,そうでない時に0をとるダミー変数,$\alpha_{i}$は個別固定効果,$\pi_{t}$は時間固定効果,$\varepsilon_{it}$は誤差項である.
\subsubsection{乗降客数交互作用分析}
乗降客数交互作用分析では,マーケットアクセスが地価に与える影響の,各地価観測点の最寄りの駅の乗降客数による変化について分析を行う.乗降客数は,駅の規模を表すデータとして用いている.乗降客数交互作用分析での推定には式\ref{interaction_passengers}に示される双方向固定効果モデルを用いた
\begin{equation}
  \ln{Y_{it}} = \alpha_{i} + \pi_{t} + \beta\ln{MA_{it}} + \beta^{pas}Z^{pas}_{i}\ln{MA_{it}} + \varepsilon_{it}
  \label{interaction_passengers}
\end{equation}
ここで,$Y_{it}$は地価観測点$i$の時点$t$における地価,$\alpha_{i}$は個別固定効果,$\pi_{t}$は時間固定効果,$MA_{it}$は地価観測点$i$の時点$t$におけるマーケットアクセス,$Z^{pas}_{i}$は地価観測点$i$の最寄り駅の乗降客数が5万人以下のときに1,それ以外のときに0をとるダミー変数,$\varepsilon_{it}$は誤差項である.

この分析において$\beta$は乗降客数が5万人を超えるときのマーケットアクセスが地価に与える効果,$\beta^{pas}$は乗降客数が5万人以下のときと,5万人を超えるときとのマーケットアクセスが地価に与える効果の差を表す.

\subsubsection{最寄り駅距離交互作用分析}
最寄り駅距離交互作用分析では,マーケットアクセスが地価に与える影響の,各地価観測点から最寄りの駅までの距離による変化について分析を行う.最寄り駅距離交互作用分析での推定には式\ref{interaction_disstation}に示される双方向固定効果モデルを用いた
\begin{equation}
  \ln{Y_{it}} = \alpha_{i} + \pi_{t} + \beta\ln{MA_{it}} + \beta^{sta}Z^{sta}_{i}\ln{MA_{it}} + \varepsilon_{it}
  \label{interaction_disstation}
\end{equation}
ここで,$Y_{it}$は地価観測点$i$の時点$t$における地価,$\alpha_{i}$は個別固定効果,$\pi_{t}$は時間固定効果,$MA_{it}$は地価観測点$i$の時点$t$におけるマーケットアクセス,$Z^{sta}_{i}$は地価観測点$i$の最寄り駅までの距離が2.5km以下のときに1,それ以外のときに0をとるダミー変数,$\varepsilon_{it}$は誤差項である.

この分析において$\beta$は最寄りの駅までの距離が2.5kmを超えるときのマーケットアクセスが地価に与える効果,$\beta^{sta}$は最寄りの駅までの距離が2.5km以下のときと,2.5kmを超えるときとのマーケットアクセスが地価に与える効果の差を表す.
\subsubsection{IC距離・最寄り駅距離交互作用分析}
IC距離・最寄り駅距離交互作用分析では,マーケットアクセスが地価に与える影響の,各地価観測地点の最寄りICまでの距離と最寄り駅までの距離の両方による変化について分析を行う.ICからの距離と最寄り駅からの距離は,それぞれ高速道路と鉄道へのアクセスのしやすさを表しており,マーケットアクセスが地価に与える効果に異なる影響を及ぼすと考えられる.ICと最寄り駅の両方のダミーを用いることで,ICまでの距離と最寄り駅までの距離を別々にコントロールすることを目指す.IC・最寄り駅距離交互作用分析での推定には,式\ref{interaction_disIC_disstation}に示される双方向固定効果モデルを用いた.
\begin{equation}
  \ln{Y_{it}} = \alpha_{i} + \pi_{t} + \beta\ln{MA_{it}} + \beta^{IC}Z^{IC}_{i}\ln{MA_{it}} + \beta^{sta}Z^{sta}_{i}\ln{MA_{it}} + \varepsilon_{it}
  \label{interaction_disIC_disstation}
\end{equation}
ここで,$Y_{it}$は地価観測点$i$の時点$t$における地価,$Z^{sta}_{i}$は地価観測点$i$の最寄り駅までの距離が2.5km以下のときに1,それ以外のときに0をとるダミー変数,$\alpha_{i}$は個別固定効果,$\pi_{t}$は時間固定効果,$MA_{it}$は地価観測点$i$の時点$t$におけるマーケットアクセス,$Z^{IC}_{i}$は地価観測点$i$の最寄りICまでの距離が5km以下のときに1,それ以外のときに0をとるダミー変数,$\varepsilon_{it}$は誤差項である.

この分析において$\beta$は最寄りの圏央道ICまでの距離が5kmを超えるかつ,最寄りの駅までの距離が2.5kmを超えるときのマーケットアクセスが地価に与える効果,$\beta^{IC}$は最寄りの圏央道ICまでの距離が5km以下のときと,5kmを超えるときとのマーケットアクセスが地価に与える効果の差,$\beta^{sta}$は最寄りの駅までの距離が2.5km以下のときと,2.5kmを超えるときとのマーケットアクセスが地価に与える効果の差を表す.
% \subsection{トレンド分析}
% トレンド分析では,双方向固定効果モデルに個別トレンド項を追加したモデルを用いた.基本分析では時間を通じて一定であるが個体ごとに異なる未観測因子である個別固定効果や,個体間では等しいが時間と共に変化する未観測因子である時点固定効果を調整して圏央道の整備効果を推定していたが,そこに加えて個別トレンド分析では時間と共に一定のトレンドを持っているが,そのトレンドの度合いが個体ごとに異なる未観測因子である個別トレンドを調整して圏央道の整備効果を推定することを目的としている.

% 式\ref{4.6}は式\ref{4.2}の双方向固定効果モデルに個別トレンド項$\lambda_{i}t$を加えたモデルである.
% \begin{eqnarray}
%   Y_{it}= \alpha_{i} + \pi_{t} +\lambda_{i}t + \beta_{1}X_{it} + \cdots  + \beta_{k}X_{kit} + \varepsilon_{it}
% \label{4.6} 
% \end{eqnarray}
% ここで,$\lambda_{i}$は個体ごとに異なる係数,$t$はトレンド変数である.個別トレンドを含んだ双方向固定効果モデルを推定する際は,ダミー変数を加えLSDV推定を用いて式\ref{4.7}を最小二乗推定する.
% \begin{eqnarray}
%   Y_{it} = &&\beta_{1}X_{1it} + \cdots  + \beta_{k}X_{kit} + \lambda_{1}I_{it}t+ \cdots  + \lambda_{N}I_{it}t \nonumber \\
%     &&+ \gamma_{1}I_{1i} + \cdots + \gamma_{N}I_{Ni} + \delta_{1}D_{1t} + \cdots + \delta_{T}D_{Tt} + \varepsilon_{it}
%   \label{4.7}
% \end{eqnarray}
% それぞれの$i=1,2,...,N$および$t=1,2,...,T$について,$Y_{it}$は被説明変数であり,($X_{1it},X_{2it},...,X_{kit}$)は説明変数を示す.
% また,$I_{1i},...,I_{Ni}$は個体$i$ごとのダミー変数,$D_{1t},...,D_{Tt}$は時点$t$ごとのダミー変数,$\varepsilon_{it}$は誤差項を示す.実際の計算では双方向固定効果モデルと同じく多重共線性を回避するためダミー変数の中からどれか一つを削除する必要があり,通常は統計ソフトがどれかを自動的に削除する.
% \vskip\baselineskip
% 以降の個別トレンド分析はマーケットアクセス分析を基本の形としており,個別トレンドが観測される可能性のある観点から分析を行う.
% \subsubsection{IC距離トレンド分析}
% 距離個別トレンド分析ではICからの距離による個別トレンドをモデルに加えて圏央道の整備効果の推定を行う.基本分析では各観測点の最寄りのICが接続している放射道路の本数や,マーケットアクセスを用いて整備効果の推定を行ったが,各観測点からICまでの距離は考慮されていない.図\ref{deruta_landprice disIC_m}からは,ICからの距離の違いによって地価の変化のトレンドはほぼ変化していないと読み取れるが,僅かな個別トレンドも調整することで,より正確な圏央道の整備効果を推定できると考える.LSDV推定を行う際のICまでの距離による個別トレンドを調整した双方向固定効果モデルは式\ref{4.8}で表される.
% \begin{eqnarray}
%   \log{Y_{it}}= &&\beta_{MA}\log{{MA}_{it}}+ \gamma_{1}I_{1i} + \cdots + \gamma_{N}I_{Ni} + \lambda_{1}d_{i}I_{it}t+ \cdots  + \lambda_{N}d_{i}I_{it}t \nonumber \\
%     && + \delta_{2001}D_{2001t} + \cdots + \delta_{2023}D_{2023t} + \varepsilon_{it}
%   \label{4.8}
% \end{eqnarray}
% ここで,$Y_{it}$は地価観測点$i$の時点$t$における地価,$MA_{it}$は地価観測点$i$の時点$t$におけるマーケットアクセス,$d_{i}$は地価観測点$i$の最寄りICまでの距離,$I_{1i},...,I_{Ni}$は各個体ダミー,$D_{2001t},...,D_{2023t}$は2001年から2023年の各年次ダミー,$\varepsilon_{it}$は誤差項である.

% \subsubsection{初期地価トレンド分析}
% 初期地価個別トレンド分析では2001年における各観測点の地価による個別トレンドをモデルに加え圏央道の推定を行う.図\ref{deruta_landprice_H13_landprice}からは,2001年時点の地価が高いと地価が大きく減るというトレンドを読み取ることができる.よって2001年時点での地価の大きさによって地価の変化のトレンドが異なることを考慮して整備効果を推定する必要があると考えられる.LSDV推定を行う際の2001年時程の地価による個別トレンドを調整した双方向固定効果モデルは式\ref{4.9}で表される.
% \begin{eqnarray}
%   \log{Y_{it}}= &&\beta_{MA}\log{{MA}_{it}}+ \gamma_{1}I_{1i} + \cdots + \gamma_{N}I_{Ni} + \lambda_{1}L_{i}I_{it}t+ \cdots  + \lambda_{N}L_{i}I_{it}t \nonumber \\
%     && + \delta_{2001}D_{2001t} + \cdots + \delta_{2023}D_{2023t} + \varepsilon_{it}
%   \label{4.9}
% \end{eqnarray}
% ここで,$Y_{it}$は地価観測点$i$の時点$t$における地価,$MA_{it}$は地価観測点$i$の時点$t$におけるマーケットアクセス,$L_{i}$は地価観測点$i$の最寄りICまでの距離,$I_{1i},...,I_{Ni}$は各個体ダミー,$D_{2001t},...,D_{2023t}$は2001年から2023年の各年次ダミー,$\varepsilon_{it}$は誤差項である

