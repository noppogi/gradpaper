\chapter{分析}

\section{固定効果推定}
固定効果モデルはパネルデータを用いた因果推論に広く用いられているモデルである.固定効果モデルは,同一個体の時間的な変動に着目することで個体が持つ個体間で異なる時間不変の未観測因子を個別固定効果として調整することを得意とする.個別固定効果項を調整することで欠落変数バイアスが生じることを防ぐことができる.固定効果モデルは式\ref{4.1}で示される回帰モデルである.
\begin{equation}
  Y_{it} = \alpha_{i} + \beta_{1}X_{1it} + \beta_{2}X_{2it} + \cdots  + \beta_{k}X_{kit} + \varepsilon_{it}
  \label{4.1}
\end{equation}
それぞれの$i=1,2,...,N$および$t=1,2,...,T$について,$Y_{it}$は被説明変数であり,($X_{1it},X_{2it},...,X_{kit}$)は説明変数を示す.
また,$\alpha_{i}$は個体$i$の時間不変の固定効果,すなわち個別固定効果を,$\varepsilon_{it}$は誤差項を示す.
個別固定効果モデルでは欠落変数バイアスをもたらす効果は,時間を通じて一定であると仮定する.

式\ref{4.1}は個体のみの固定効果を考慮したモデルであるが,個体に加えて時間の固定効果を考慮した固定効果モデルも存在する.時間固定効果とは個体間では等しいが時間と共に変化する未観測因子のことを指し,個体固定効果と時間固定効果を同時に調整する固定効果モデルは双方向固定効果モデルと呼ばれる.双方向固定効果モデルは式で示される回帰モデルである.
\begin{equation}
  Y_{it} = \alpha_{i} + \pi_{t} + \beta_{1}X_{1it} + \beta_{2}X_{2it} + \cdots  + \beta_{k}X_{kit} + \varepsilon_{it}
  \label{4.2}
\end{equation}
それぞれの$i=1,2,...,N$および$t=1,2,...,T$について,$Y_{it}$は被説明変数であり,($X_{1it},X_{2it},...,X_{kit}$)は説明変数を示す.
また,$\alpha_{i}$は個体$i$の個別固定効果を,$\pi_{t}$は時間$t$における時間固定効果を,$\varepsilon_{it}$は誤差項を示す.

固定効果モデルを用いる場合は,個別固定効果$\alpha_{i}$が説明変数と相関している状況を考えるため,通常の最小二乗法は不適切である.そこで,固定効果推定と呼ばれる手法を用いる.固定効果推定は2つの段階からなる.第1段階において固定効果をモデルから除去し,堕2段階で固定効果が除去されたモデルを最小二乗推定する.
\section{分析手法}

「あなたのこと全く好きではないけど、付き合ってもいいわ。その代わりに、わたしをちゃんと守ってね。理想として、あなたが死んでもいいから」彼女に告白し、そして奇妙な条件付きの返事をもらった瞬間から、僕は彼女の為に生きはじめた。この状況が僕に回ってきたことが、神様からの贈り物であるようにも思える。この結果が、いつの日か、遠い遠い全く別の物語に生まれ変わりますように。入間人間の名作が、『六百六十円の事情』 『ぼっちーズ』 でコンビを組んだ宇木敦哉のイラストによって、待望の文庫化!
\subsection{基本分析}
\subsubsection{接続本数分析}

「あなたのこと全く好きではないけど、付き合ってもいいわ。その代わりに、わたしをちゃんと守ってね。理想として、あなたが死んでもいいから」彼女に告白し、そして奇妙な条件付きの返事をもらった瞬間から、僕は彼女の為に生きはじめた。この状況が僕に回ってきたことが、神様からの贈り物であるようにも思える。この結果が、いつの日か、遠い遠い全く別の物語に生まれ変わりますように。入間人間の名作が、『六百六十円の事情』 『ぼっちーズ』 でコンビを組んだ宇木敦哉のイラストによって、待望の文庫化!

\subsubsection{所要時間最小分析}

「あなたのこと全く好きではないけど、付き合ってもいいわ。その代わりに、わたしをちゃんと守ってね。理想として、あなたが死んでもいいから」彼女に告白し、そして奇妙な条件付きの返事をもらった瞬間から、僕は彼女の為に生きはじめた。この状況が僕に回ってきたことが、神様からの贈り物であるようにも思える。この結果が、いつの日か、遠い遠い全く別の物語に生まれ変わりますように。入間人間の名作が、『六百六十円の事情』 『ぼっちーズ』 でコンビを組んだ宇木敦哉のイラストによって、待望の文庫化!

\subsection{個別トレンド分析}
\subsubsection{距離個別トレンド分析}
\subsubsection{初期地価個別トレンド分析}

「あなたのこと全く好きではないけど、付き合ってもいいわ。その代わりに、わたしをちゃんと守ってね。理想として、あなたが死んでもいいから」彼女に告白し、そして奇妙な条件付きの返事をもらった瞬間から、僕は彼女の為に生きはじめた。この状況が僕に回ってきたことが、神様からの贈り物であるようにも思える。この結果が、いつの日か、遠い遠い全く別の物語に生まれ変わりますように。入間人間の名作が、『六百六十円の事情』 『ぼっちーズ』 でコンビを組んだ宇木敦哉のイラストによって、待望の文庫化!

